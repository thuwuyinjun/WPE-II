\documentclass{article}
\usepackage[utf8]{inputenc}
\usepackage{setspace}
\usepackage[left=2cm, right=2cm, top=2cm]{geometry}
\usepackage{graphicx}
\usepackage{amsmath}
\usepackage{amsthm}
\usepackage{mathtools}
\usepackage{amssymb}
\usepackage{tikz}
\usepackage{algpseudocode}
\usepackage{amsmath}
\usepackage{multirow}
\usepackage{hhline}
\usepackage[linesnumbered,ruled]{algorithm2e}

\usetikzlibrary{%
  arrows,
  calc
}
\theoremstyle{definition}
\newtheorem{example}{Example}[section]
\DeclarePairedDelimiter\ceil{\lceil}{\rceil}
\DeclarePairedDelimiter\floor{\lfloor}{\rfloor}
\usetikzlibrary{backgrounds}
\usetikzlibrary{calc}
\usetikzlibrary{positioning}
\usetikzlibrary{shapes}
\newcommand{\eat}[1]{}
\newenvironment{proof}{\paragraph{Proof:}}{\hfill$\square$}


\title{Toward efficient incremental computations in data analytics: a review of recent research}
\author{wuyinjun }
\date{}
\onespacing

\begin{document}

\maketitle

\begin{abstract}
    
\end{abstract}




\section{Introduction}

%problem description
Machine learning algorithms have been broadly applied in various data analysis applications in recent years, which pose new challenges in various aspects. One of such challenge is that the collections of data for training or prediction are not only huge but also constantly updated while the real-time performance is expected in many applications. Typical examples include online clustering analysis for clickstream \cite{guha2000clustering}, incremental collaborative filtering analysis for recommendation systems \cite{papagelis2005incremental} etc. Furthermore, the analytic workloads for building machine learning models or making predictions over the data may also arrive in a continuous manner. Considering the possible similarity between the requests among the workloads, whether it is possible to cache and reuse those computation results so as to reduce the computation overhead is another concern for users.
% All of those applications and solutions share the same spirit, i.e. {\em computing new machine learning model incrementally without retraining the model from the scratch.}

%Database solutions for it
Similar situation happens in database community, where database instances are large and versioned and thus {\em database view} is materialized for efficient query response \cite{date2006relational}, maintained to reflect changes from the underlying base relations \cite{gupta1993maintaining, green2007update} and reused to reduce query execution overhead \cite{halevy2001answering}, which plays an important role in data integration \cite{levy1996querying}, query optimization \cite{rajaraman1995answering}, data visualization \cite{brachman1993integrated}. 

Since such view-related problem has been extensively studied from the both theoretical and systematical side in database domain, \eat{which targets at improving the query performance,} it is natural to apply those ideas to data analysis tasks by regarding some pre-computed results as view. Then two major research problems on this kind of special views have emerged in recent applications, i.e. 1) how to {\em incrementally update} the those pre-computed results in an efficient way in the case of update to the data over which those results are constructed; 2) how to {\em reuse} those pre-computed results for the following computations, which are closely related to traditional {\em incremental view maintenance problem} and {\em query rewriting using views problem} from database community respectively. which have been explored by recent works in database communities. 

Both of the two research problems can span in different lines, depending on which kinds of pre-computed results are materialized as ``views''. Considering the fact that there are two typical computation phases in typical machine learning pipelines, i.e. training phase and test phase, the views can be the result of training phase, i.e. the {\em model parameters}, or the output of test phase, i.e. the {\em classification results}. Besides, since the two computation phases are composed of a series of linear algebra operations, the idea of materializing their results can also be extended to more {\em general linear algebra programs}. 

In terms of the first research problem, materialization of the pre-computed results can appear in all of the three scenarios mentioned above. Specifically, \cite{deshpande2006mauvedb} and \cite{gupta2015processing} are on incrementally updating the materialized model parameters, \cite{koc2011incrementally} is on incrementally updating the materialized classification results while \cite{nikolic2014linview} is on incrementally updating the output of the linear algebra programs, which rely on some assumptions in different aspects. One typical assumption for those works is about the updates against the pre-computed results, in which the type of updates can be addition, removal or modifications to the input data, the amount of updates can be small or large or arbitrary and the updates can even appear in different dimensions, i.e. feature level and sample level. Table \ref{tab:assump_updates} gives an overview of all of the papers to be discussed along with the assumptions about updates that they follow. For example, for \cite{nikolic2014linview}, the updates should be small modifications to the input data rather than addition or removal of input data, which, however, can appear at both sample level and feature level. There are also some other minor assumptions for specific cases in those works. For example, in \cite{koc2011incrementally}, the goal is to incrementally update the prediction results in the test phase. So the time to update the model parameters is assumed to be negligible. Another example is in \cite{nikolic2014linview}, in which the linear algebra programs are assumed to be composed of pure linear operations without any other complex non-linear operations. More discussions will be expanded in the following sections.

\begin{table}[h]
    \centering
    \begin{tabular}{|c|c|c|c|}\hline
        % && \multicolumn{3}{|c|}{Eager {\em Update} (updates/s)}&\multicolumn{3}{|c|}{Lazy {\em All members} (scan/s)}  \\\hhline{~~------}
        % &&Re-evaluation & Incremental\\ \hline
        &&Sample level& Feature level\\\hline
    \multirow{2}{*}{amount of updates}&small&\cite{deshpande2006mauvedb, gupta2015processing,koc2011incrementally, nikolic2014linview}  & \cite{nikolic2014linview} \\\hhline{~---}
    &arbitrary&\makecell{linear regression in \cite{deshpande2006mauvedb, gupta2015processing}\\ naive bayes in \cite{gupta2015processing}} & \\\hline
    % \multicolumn{2}{|c|}{Hybrid}&2.0 & 6.6 & 0.2 & 8.0 & 48.8 & 2.1\\\hline
    \multirow{3}{*}{type of updates}&addition of data&\cite{deshpande2006mauvedb, gupta2015processing,koc2011incrementally} &  \\\hhline{~---}
    &removal of data&\makecell{linear regression in \cite{deshpande2006mauvedb, gupta2015processing}\\ naive bayes in \cite{gupta2015processing}} &  \\\hhline{~---}
    &modifications to data&\cite{nikolic2014linview} & \cite{nikolic2014linview} \\\hline
    % \multirow{3}{*}{Space}&Linear & $n^2$ & $n^2k$\\\hhline{~---}
    % &Exponential & $n^2$ & $n^2log k$\\\hhline{~---}
    % &Skip-s & $n^2$ & $n^2(logs+\frac{k}{s})$\\\hline
    \end{tabular}
    \caption{Assumptions about updates}
    \label{tab:assump_updates}
\end{table}

For the second research problem, the pre-computed results to be dealt with are only the machine learning models, which is primarily explored by \cite{gupta2015processing} and \cite{hasani2018efficient}. In those two papers, the pre-computed machine learning models are reused to construct approximated models for user's requests in two different manners, i.e., merging pre-computed model parameters directly and combining the coresets \cite{agarwal2005geometric} of the pre-computed model parameters, for the purpose of saving the computation time. In order to avoid significant differences between the constructed models and the models built from the scratch, there should exist some approximation rate for combined models, which depends on the properties of specific type of machine learning models, such as the approximation properties on coresets \cite{agarwal2005geometric}. It indicates that only for the models with such approximation properties such as K-means, Gaussian mixture models, SVM and logistic regression, the solutions from \cite{gupta2015processing} are applicable. 

Besides, the second research problem is also coherent to incremental model parameters update problem as mentioned before. For those types of machine learning models over which the updates can be removal of data (see Table \ref{tab:assump_updates}), when combining the pre-computed machine learning models, their effect can be either accumulated or canceled while for other types of models introduced in \cite{deshpande2006mauvedb, gupta2015processing,koc2011incrementally}, the combinations of models only means the ``addition'' of the effect of those model. More details are presented in the following sections.

This article is organized as follow, in Section \ref{sec: pre}, some basic notations and concepts for various machine learning algorithms are introduced, which prepares for the following discussions for incremental maintenance for pre-computed results in Section \ref{sec: maintain_views} and reuse of pre-computed results in Section \ref{sec: view_reuse} in the context of data analytics.
\section{Preliminaries}\label{sec: pre}
We will use capitalized letters (e.g. $H$) to denote matrix, capitalized bold letters (e.g. $\textbf{T}$) to denote tables, lowercase letters (e.g. $x$ and $y$) to denote scalar value, bold lowercase letters (e.g. $\textbf{x}$) to denote variables or attributes in the table, letters with bar (e.g. $\bar{w}$) to denote a vector, $f(*)$ to denote functions ($*$ represents arguments of function $f$). Besides, we use $H_{i,j}$ to denote the value at cell $(i,j)$ in matrix $H$ and use $\bar{w}_i$ to denote $i_{th}$ element in vector $\bar{w}$. We introduce superscript $^{(t)}$ (e.g. $x^{(t)}$) to denote the value of certain variables at the time step $t$ during iterative computation. For a tuple $t$ in a table $\textbf{T}$, we use $t.attr$ to denote the value of attribute $attr$ in tuple $t$ where $attr$ is in the schema of $\textbf{T}$. Based on those notations, we provide notations for each model as follows.

\paragraph{Regression model}
A regression model is usually written as the following form:
\begin{equation}
\textbf{y}=\sum_{i=1}^kw_ih_i(*)
\end{equation}
where $h_i$ represents {\em basis function}, which is a monomial of {\em predictor variables} while $w_i$ represents coefficient of those monomials and $\textbf{y}$ is the {\em response variable}.

For example, in wireless sensor network application, temperature $\textbf{t}$ is measured at a 2D-space with coordinate $\textbf{x}_1$ and $\textbf{x}_2$, which can be computed with the following typical polynomial of the two predictors variables:

\begin{equation}
\textbf{t}=w_1 + w_2\textbf{x}_1+w_3\textbf{x}_1^2 + w_4\textbf{x}_2+w_5\textbf{x}_2^2
\end{equation}
where the basis functions are $\{h_1(\textbf{x}_1,\textbf{x}_2), h_2(\textbf{x}_1,\textbf{x}_2), h_3(\textbf{x}_1,\textbf{x}_2),h_4(\textbf{x}_1,\textbf{x}_2), h_5(\textbf{x}_1,\textbf{x}_2)\}=\{1, \textbf{x}_1, \textbf{x}_1^2, \textbf{x}_2, \textbf{x}_2^2\}$

Given a set of data\eat{ observed from the underlying wireless sensors located at different positions}, $\{\bar{x}_j,y_j\}(j=1,2,\dots,n)$, where $\bar{x}_j=\{x_{j1},x_{j2}, \dots, x_{jr}\}$, the coefficients $\bar{w}^* = \{w_1,w_2,\dots,w_k\}$ are estimated with the following linear system:
\begin{equation}\label{eq: regression_solve}
    H^TH\bar{w}^*=H^T\bar{y}
\end{equation}

thus $\bar{w}^*$ is computed as:
\begin{equation}\label{eq: regression_solve_final}
    \bar{w}^*=(H^TH)^{-1}H^T\bar{y}
\end{equation}



where $H$ is:
\begin{equation}
    H=\begin{bmatrix}
h_1(\bar{x}_1) & h_2(\bar{x}_1) &\dots &h_k(\bar{x}_1)\\
h_1(\bar{x}_2) & h_2(\bar{x}_2) &\dots &h_k(\bar{x}_2)\\
\dots\\
h_1(\bar{x}_n) & h_2(\bar{x}_n) &\dots &h_k(\bar{x}_n)
\end{bmatrix}
\end{equation}

and $\bar{y}$ is:

\begin{equation}
    \bar{y} = \{y_1, y_2,\dots, y_n\}^T
\end{equation}

In general, $H$ will be a matrix which represents data sets of $n$ data points with $k$ features while $\bar{y}$ represents the label vectors. But in practice, usually the simple case is considered where the exponent of the predictor variables is 1 and there is no cross terms, which means that $h_j(\bar{x}_i) = x_{ij}$ and thus $H$ becomes:

\begin{equation}
    H=\begin{bmatrix}
x_{11} & x_{12} &\dots &x_{1k}\\
x_{21} & x_{22} &\dots &x_{2k}\\
\dots\\
x_{n1} & x_{n2} &\dots &x_{nk}\\
\end{bmatrix}
=\begin{bmatrix}
\bar{x}_1\\
\bar{x}_2\\
\dots\\
\bar{x}_n\\
\end{bmatrix}
\end{equation}

If we represent $X = \begin{bmatrix}
x_{11} & x_{12} &\dots &x_{1k}\\
x_{21} & x_{22} &\dots &x_{2k}\\
\dots\\
x_{n1} & x_{n2} &\dots &x_{nk}\\
\end{bmatrix}$, then Equation \ref{eq: regression_solve_final} can be rewritten as:

\begin{equation}\label{eq: regression_solve_simple_final}
    \bar{w}^*=(X^TX)^{-1}X^T\bar{y}
\end{equation}


\paragraph{Interpolation model} 

The goal of interpolation is to estimate missing values of response variables given values of the predictor variables, which does not exist in the set of existing value pair for predictor variable and response variable. Specifically, given a variable pair $(\textbf{t},\textbf{v})$ and a set of observations for those two variables, $(t_i, v_i)(i=1,2,\dots,n)$ are then used to estimate value $v'$ of variable $\textbf{v}$ given a value $t'$ of variable $\textbf{t}$ $(t_j< t' < t_{j+1})$. Usually, {\em linear interpolation} is used. So $v'$ is estimated with value pair $(t_j, v_j)$ and $(t_{j+1}, v_{j+1})$ as follows:
\begin{equation}\label{eq: interpolation}
    v'= v_j + (v_{j+1}-v_j)\times\frac{t'-t_j}{t_{j+1}-t_j}
\end{equation}



\paragraph{Logistic regression}
Logistic regression is a typical linear classifier, which computes the probability of the membership of the data points with logistic function but has linear decision boundary. In the case of binary classification, the model uses the following function to determine the class probability:

\begin{equation}\label{eq: logistic_regression_prob}
    P(\textbf{y}=y|\bar{x}) = \frac{1}{1+e^{-y\bar{w}^T\bar{x}}}
\end{equation}

where $y \in \{-1,1\}$.

To identify the model parameter $\bar{w}^T$ given a set of training data points, $\{(\bar{x}_i, y_i)\}_{i=1}^n$, the following loss function needs to be optimized:
\begin{equation}\label{eq: logistic_objective_function}
    F(\bar{w}) = \frac{1}{n}\Sigma_{i=1}^nL(\bar{w};\bar{x}_i, y_i) + \lambda R(\bar{w})
\end{equation}

where $L$ is the loss function, which is usually the cross entropy loss function as below:
\begin{equation}\label{eq: logistic_loss_function}
    L(\bar{w}; \bar{x}_i, y_i) = y_ilogP(\textbf{y}=y_i|\bar{x}_i) + (1-y_i)log(1-P(\textbf{y}=y_i|\bar{x}_i))
\end{equation}

and $R(\bar{w})$ is the regularization term, which is usually $L2$ term:

\begin{equation}
    R(\bar{w}) = \bar{w}^T\bar{w}
\end{equation}


By combining Equation \ref{eq: logistic_loss_function}, the loss function in Equation \ref{eq: logistic_objective_function} targets at finding the model parameter $\bar{w}*$ such that the likelihood of the training data set is maximized. 

However, unlike other models such as linear regression, there is closed form of the optimal solution for Equation \ref{eq: logistic_objective_function}, which is thus solved by Stochastic Gradient Descent (SGD) algorithm \cite{robert2014machine} iteratively until it reaches convergence. That is, the parameter $\bar{w}$ is updated as:

\begin{equation}
    \bar{w} \leftarrow \bar{w} - \alpha \triangledown F_i(\bar{w})
\end{equation}

where $\triangledown F_i(\bar{w})$ is computed by simply using $i_{th}$ data point for the gradient of the loss function:
\begin{equation}
    \triangledown F_i(\bar{w}) = \triangledown L(\bar{w};\bar{x}_i,y_i) + \lambda \bar{w} = y_i\bar{x}_i(1-P(\textbf{y}=y_i|\bar{x}_i)) + \lambda \bar{w}
\end{equation}

\paragraph{Naive Bayes} Naive Bayes is a simple probabilistic model, which also computes the membership probability of a data point $\bar{x} = \{x_1, x_2, \dots, x_d\}$ by Bayes theorem:

\begin{equation}\label{eq: bayes_theorem}
P(\textbf{y}=c|\bar{x}) = \frac{P(\textbf{y}=c)P(\bar{x}|\textbf{y}=c)}{P(\bar{x})}    
\end{equation}

Under the assumption that the features in $\bar{x}$ are independent from each other, Equation \ref{eq: bayes_theorem} can be rewritten as follows (without caring about the appearance probability of $\bar{x}$):

\begin{equation}\label{eq: nb_exp}
    P(\textbf{y}=c|\bar{x}) \propto P(\textbf{y}=c)\Pi_{i=1}^dP(x_i|\textbf{y}=c)
\end{equation}

in which $P(\textbf{y}=c)$ is usually estimated by the frequency of class $c$. Suppose the size of training data sets is $N$ and the number of data points in class $c$ is $N_c$, then $P(\textbf{y}=c) = \frac{N_c}{N}$.

In practice, for each class $c$, the calculation of $P(x_i|\textbf{y}=c)$ depends on the distribution assumptions over the underlying data. Usually, Gaussian distribution is preferred and thus $P(\bar{x}|\textbf{y}=c)$ is written as:

\begin{equation}\label{eq: nb_guassian}
    P(\bar{x}|\textbf{y}=c) = \Pi_{i=1}^dP(x_i|\textbf{y}=c) = \Pi_{i=1}^dN(x_i|\mu_{jc}, \sigma_{jc}^2)
\end{equation}

in which $\mu_{jc}$ and $\sigma_{jc}$ are the mean and variance of the $j_{th}$ feature for the data point belonging class $c$. The two parameters can be computed as below given a set of training data set $\{(\bar{x}_i, y_i)\}$($i=1,2,\dots, n$):

\begin{equation}\label{eq: nb_mean}
    \mu_{jc} = \frac{\Sigma_{i=1}^nx_{ij}[y_i=c]}{N_c}
\end{equation}
\begin{equation}\label{eq: nb_var}
    \sigma_{jc}^2 = \frac{\Sigma_{i=1}^n(x_{ij}[y_i=c])^2}{N_c}-(\frac{\Sigma_{i=1}^nx_{ij}[y_i=c]}{N_c})^2
\end{equation}

in which $x_{ij}$ represents the value of $j_{th}$ feature from the $i_{th}$ data point and $[y_i=c]$ is an identity function depending on whether the $i_{th}$ data point is in class $c$ or not (it is evaluated to 1 if $y_i$ is $c$ otherwise 0).

\paragraph{K-means} K-means is an unsupervised clustering algorithm, which aims at computing a set of centroids $C$ from a set of data points $\{\bar{x}_i\}$($i=1,2\dots,n$) and assigning each data point from $\{\bar{x}_i\}$ to one of centroid in $C$ by minimizing the sum of square errors (SSE) with a similarity measure $d(*)$ in it:

\begin{equation}\label{eq: sse_k_means}
    SSE = \sum_{i=1}^nd(\bar{x}_i, C)
\end{equation}

\paragraph{GMM} GMM is the probabilistic version of K-means, which is parameterized by a set of model parameters $\bar{\theta} = \{(w_1, \bar{\mu_1}, \Sigma_1), (w_2, \bar{\mu_2}, \Sigma_2), \dots, (w_k, \bar{\mu_k}, \Sigma_k)\}$ such that the probability of a data point $\bar{x}$ belonging to a cluster $i$ is computed by the probability density function of normal distribution and weighted by $w_i$, i.e.  $w_i\frac{exp(-\frac{1}{2}(\bar{x}-\bar{\mu}_i)^T\Sigma^{-1}(\bar{x}-\bar{\mu}_i))}{\sqrt{(2\pi)^k|\Sigma|}}$. The model parameters are derived during the training process with EM algorithm.

\paragraph{GLM} GLM includes a large class of typical linear classifiers such as logistic regression and support vector machine (SVM) and also typical regression methods, such as linear regression.
\section{Incrementally maintain views in data analysis tasks}\label{sec: maintain_views}
%road map
In this section, some recent works on incrementally maintaining views in the data analysis environment are presented, in which the views can be machine learning model parameters \cite{deshpande2006mauvedb, gupta2015processing}, classification results by applying machine learning models \cite{koc2011incrementally} or the output of the general linear algebra programs \cite{nikolic2014linview}. Those works are introduced in the following three subsections respectively. \eat{The first one is to maintain materialized model parameters in database, which initially originates from  and follows by some other papers such as  while the other one is to maintain the classification results by the model, which is from .}

% develop a system (MauveDB vs Hazy) to store typical statistical model parameters as database view in RDBMS and update the views incrementally when raw data are modified.   

\subsection{Incrementally maintain model parameters}\label{sec: maintain_model_para}
%brief description to MauveDB
In this subsection, \cite{deshpande2006mauvedb} and parts of \cite{gupta2015processing} are presented, which regard the machine learning model parameters as database views and handle the changes over those models incrementally. In \cite{deshpande2006mauvedb}, the authors developed a system called MauveDB, which primarily deals with wireless sensor network applications, where data are collected through underlying sensors, trained with typical machine learning models such as regression-based model and interpolation-based model and used for estimating missing or future results. 
% The statistic models learned with those raw data are materialized as {\em model-based views}.
% , which are created with a high-level SQL-like language. 
The raw data from the underlying sensors are changing constantly, which requires efficient view maintenance strategies. 
Compared to \cite{deshpande2006mauvedb}, parts of \cite{gupta2015processing} also deals with the same problem but richer types of machine learning models, which, ultimately, goes towards the reuse of models in analytical workloads. In general, there are multiple ways to maintain the model parameters once the model is constructed, e.g. always computed from the scratch. Different model maintenance strategies introduced in \cite{deshpande2006mauvedb} and the details on incremental maintenance of model parameters for each machine learning algorithms are provided as follows.
% which highly depend on the types of statistical model.



\eat{\subsubsection{View definition}

MauveDB developed an declarative SQL-like language for users to define model-based view, which is exemplified in Figure \ref{fig:MauveDB_view_def}. Although different models cannot be manipulated in exactly the same way due to their different characteristics, the commonalities are still leveraged in this language. 

In Figure \ref{fig:MauveDB_view_def}, examples of regression-based view and Interpolation-based view definitions are presented in (i) and (ii) respectively. Same as way to create other database views, the view schemas, where the raw data comes from to construct the views (represented by \textit{SELECT}, \textit{FROM}) and what conditions the raw data should satisfy (represented by \textit{WHERE}) should be declared in the view definition statements. The model types (\textit{FIT} and \textit{INTERPOLATE} for regression and interpolation respectively) should be also specified along with partitions of raw data on each of which the models are trained (\textit{FOR EACH}) and other model-related information (such as base function for regression model, represented by \textit{BASE}). After executing statements shown in Figure \ref{fig:MauveDB_view_def}, {\em model-based views} are then created and stored in the RDBMS as relational tables.


\begin{figure}
    \centering
    \includegraphics[width=8cm, height=8cm]{Figures/MauveDB_create_view.png}
    \caption{Define views in MauveDB}
    \label{fig:MauveDB_view_def}
\end{figure}
}
\subsubsection{View maintenance strategies}
%High-level descriptions of view maintenance strategies
When new data are available, the updates should be reflected on the model-based views. Four different strategies are developed in MauveDB for this purpose and their trade-offs are discussed with extensive experiments. The details of the four view maintenance strategies are introduced as below.

\textit{Strategy 1: Materialize the views}
A naive way is simply to construct the models and materialize the model parameters as views in the database, which can avoid unnecessary query execution time when users want to retrieve the ``exact'' model information. However, such views may incur high space overhead and pose a challenge in refreshing the model as new data comes.

\textit{Strategy 2: Always use the base data }
This strategy goes to another extreme without materialization at all in the database, which can obviously make query processing very expensive since recomputating the view content is necessary every time when queries are evaluated. 

\textit{Strategy 3: Partial materialization}
Somewhere in-between is to partially materialize the view content, which simply caches parts of the views that have been computed by certain queries. When the views are about to be refreshed, the corresponding cached contents in memory will be invalidated.

\textit{Strategy 4: Materialize an intermediate representation}
Some intermediate representations are materialized in this strategy by leveraging some nice properties of the machine learning models, which have been proven to be an efficient solution by experiments and are presented below for each model respectively.

The specific choice of strategies in practice depends on various factors, such as the query workload, data statistics and types of models. The trade-off between the four strategies are experimentally explored with extensive experiments in \cite{deshpande2006mauvedb}, which shows that the strategy 4 performs best in most scenarios.

\subsubsection{View maintenance for each model}\label{sec: view_maintenance_model}
\paragraph{view maintenance for regression model} \cite{deshpande2006mauvedb} and \cite{gupta2015processing} apply strategy 4 over the regression model in the same way. Recall that the optimal coefficients can be solved with Equation \ref{eq: regression_solve}. The intermediate representation for it can be simply the materialization of matrix $H^TH$ and $H^T\bar{y}$, which can be beneficial in various aspects. First, the dimensions of the two matrices only rely on the total number of features, which thus won't lead to large space overhead. Besides, efficient updates to the two matrices are achievable since they are computed with linear operators, i.e. matrix multiplications and additions. A newly generated data point $(\bar{x}',\bar{y}')$ can trigger the update of $H^TH$ and $H^TY$ as follows:
\begin{equation}
    (H^TH)^{new}_{i,j} = (H^TH)^{old}_{i,j} + h_i(\bar{x}')*h_j(\bar{x}')
\end{equation}
\begin{equation}
    (H^TY)^{new}_i = (H^TY)^{old}_i + h_i(\bar{x}')*\bar{y}'^T
\end{equation}

Where $(H^TY)^{x}_i$ ($x \in \{old, new\}$) represents the $i_{th}$ entry in the vector $(H^TY)^{x}_i$. Once $H^TH$ and $H^TY$ are updated, the coefficient $W$ is computed with Equation \ref{eq: regression_solve_final}, 
% \begin{equation}
%     \bar{w}^*=(H^TH)^{-1}H^T\bar{y}
% \end{equation}
which can be computed with time complexity $Q(k^3)$ where $k$ is the total number of basis functions. As mentioned before, in the simplest case when $H=X$, i.e. $H_i(*)$s are all identity functions and the number of basis functions is the same as the number of features, then $k$ also represents the number of features.

\paragraph{View maintenance for interpolation model} \cite{deshpande2006mauvedb} also processes interpolation model by applying strategy 4, in which the data in the interpolation-based view $\textbf{V}$ will be of the form $(\textbf{t}, \textbf{v})$ where $\textbf{t}$ and $\textbf{v}$ represent predictor variable and response variable respectively. Unlike regression mode where the intermediate representations are the intermediate computation results, the intermediate representations of interpolation model are some auxiliary data structure for searching the values of variable $\textbf{t}$. When new data point $(t', v')$ is inserted into the $\textbf{V}$, then the data structure should be updated to reflect the effect of the insertion such that when predicting the value for $\textbf{v}$ given $\textbf{t}$ value $t''$, the values of $t_{i}$ and $t_{i+1}$ used in Equation \ref{eq: interpolation} for the prediction can be efficiently retrieved with the updated data structure from $\textbf{V}$.
% In order to efficiently estimate the value of $\textbf{v}$ given the value of $\textbf{t}$, $t'$, which is missing from the view instance but queried by users, the closest $\textbf{t}$ values $t_{-}$ and $t_{+}$ (along with the corresponding $\textbf{v}$ values $v_{-}$ and $v_{+}$) to $t'$ are retrieved with the auxillary data structure such that $t'$ lies in the interval $(t_{-}, t_{+})$ without any other $\textbf{t}$ value from $\textbf{V}$ inside the interval. So the estimated $\textbf{v}$ value for $t'$ can be computed using Equation \ref{eq: interpolation}.


\paragraph{View maintenance for Naive Bayes model}
As mentioned before, \cite{gupta2015processing} deals with richer types of machine learning models such as Naive Bayes model and logistic regression model, which still follows the ideas of strategy 4 to maintain the model-based views. In terms of Naive Bayes model, based on Equation \ref{eq: nb_exp} and Equation \ref{eq: nb_guassian}, the membership probability depends on the values of $n_c$, $\mu_{jc}$ and $\sigma_{jc}^2$, in which $\mu_{jc}$ and $\sigma_{jc}^2$ require the sum and square of sum of the feature values within the class $c$, i.e. $\Sigma_{i=1}^n(x_{ij}I(y_i=c))^2$ and $\Sigma_{i=1}^nx_{ij}I(y_i=c)$ (denoted as $SS_{jc}$ and $S_{jc}$ respectively) from Equation \ref{eq: nb_mean} and \ref{eq: nb_var}, which are thus materialized. When a new data point $\bar{x}, y$ comes where $\bar{x}^T = [x_1, x_2, \dots, x_d]$, $n_c$, $S_{jc}$ and $SS_{jc}$ is updated as:

\begin{equation}
    \begin{split}
        n_c' &= n_c + I(y=c)\\
        S_{jc}' &= S_{jc} + x_jI(y=c)\\
        SS_{jc}'&= SS_{jc} + x_j^2I(y=c)\\ &= SS_{jc} + (x_jI(y=c))^2
    \end{split}
\end{equation}

\paragraph{View maintenance for logistic regression model}
Unlike the machine learning models discussed above, there is no closed-form solution for logistic regression model, which requires gradient descent algorithm with non-linear operator (i.e. logistic function) involved to derive the model parameter $\bar{w}$ in Equation \ref{eq: logistic_regression_prob}. 

As the first step toward incrementally maintaining the logistic regression model, a variant of statistical gradient descent, i.e. Mixture Weight Methods \cite{mcdonald2009efficient} is used, which is sketched in Algorithm \ref{alg: mixture weight method}. Intuitively, this algorithm splits the entire dataset into $p$ partitions in the beginning and the size of each partition is $l$. For $i_{th}$ partition, the gradient descent algorithm is applied to derive the model parameter (denoted by $\bar{w}_i$). In the end, the model parameters computed from every partition are averaged as the final output. Then given the training data points $\{\bar{x}_i, y_i\}_{i=1}^n$ on which Algorithm \ref{alg: mixture weight method} is applied, the model parameter derived from each partition is materialized, which forms a parameter set $P = \{\bar{w}_1, \bar{w}_2, \dots, \bar{w}_p\}$. 

To incrementally update the model parameter, the new data points due to the updates are also split into partitions of size $l$ and the model parameter for each partition is derived by applying SGD as Algorithm \ref{alg: mixture weight method} proceeds, which is also added into $P$. In the end, the updated model parameter $\bar{w}^{new}$ is computed by averaging the parameters from the updated $P$. According to \cite{mcdonald2009efficient}, the gap between $\bar{w}^{new}$ and the model parameter computed over the updated training data set from the scratch is bounded.


\begin{algorithm}[h!] 
\footnotesize
%  \SetKwInOut{Input}{Input}
%  \SetKwInOut{Output}{Output}
%  \Input{a set of valid view mappings $\mathcal{M}$ for query tuple $t \in Q(D)$, query $Q$}

%  \Output{a set of covering sets $C$}
 Partition $\{\bar{x}_i, y_i\}$ into $p$ partitions, $\{S_1, S_2,\dots, S_p\}$, each of which is of size $l$.
 
 \For{all $i \in \{1,2,\dots, p\}$}
 {
    $\bar{w}_i \leftarrow 0$
    
    \For{$t \leftarrow 1$ to T}
    {
        \tcp{ $F_{S_i}(\bar{w}) = \frac{1}{|S_i|}\Sigma_{(\bar{x}_i, y_i) \in S_i}L(\bar{w};\bar{x}_i, y_i) + \lambda R(\bar{w})$, which is adapted from Equation \ref{eq: logistic_objective_function}}
        $\triangledown F_{S_i}(\bar{w}) \leftarrow GRADIENT(F_{S_i}(\bar{w}))$
    
        $\bar{w}_i \leftarrow \bar{w}_i + \lambda(\triangledown F_{S_i}(\bar{w}))$
    }
 }
 
 Aggregate all $\bar{w}_{\mu} = \Sigma_{k=1}^p\mu_k\bar{w}_k$    
 
 \caption{Mixture weight method}
 \label{alg: mixture weight method}
 \end{algorithm}




\subsection{Incrementally maintain classification result}\label{sec: maintain_classification}
Hazy \cite{koc2011incrementally} is a system, aiming at incrementally maintaining classification results such that the performance of the read and update operations over the classification results are optimized, which mainly supports typical linear machine learning algorithms, including support vector machines, ridge regression and logistic regressions.

\subsubsection{Views in Hazy}


\eat{\begin{figure}
    \centering
    \includegraphics[width=10cm, height=2.5cm]{Figures/Hazy_view.png}
    \caption{Define views in Hazy}
    \label{fig:Hazy_view_def}
\end{figure}}

%intro to view definition
In Hazy, the goal is to classify a set of entities (samples) with some classifier. Each entity has the form $(id, \bar{f})$, where the $id$ is the entity id while $\bar{f}$ is the feature values for this entity. \eat{users can declare views in the way shown in Figure \ref{fig:Hazy_view_def}, in which a {\em classification view} is defined over some entity table (i.e. $Papers$ in Figure \ref{fig:Hazy_view_def}, with Primary key $id$) to identify their type (i.e. $Paper\_area$) with a model trained using some training examples (i.e. $Examples\_papers$) that are derived from original entity table by applying some feature function (i.e. $tf\_bag\_of\_words$ function). So in the end, }The view content is composed of the classification results for each entity, in which each tuple has the form $(id, class)$ where $class$ is the label determined by the machine learning model for the entity with identifier $id$. There are two strategies to store the view content, i.e. {\em eager strategy} and {\em lazy strategy}. The major difference between the two strategies is that the former one stores the materialized views in the database while the latter one only keeps views virtual.

%operations in Hazy and the bottleneck for eager and lazy approach
Hazy focus on three typical {\em queries} issued by the users: 1) {\em Single entity} read, i.e. retrieve the label of a single entity; 2) {\em All members} read, i.e. retrieve the labels for all the entities; 3) {\em Update}, i.e. update the classification results in the view content once the model is updated due to some incoming new training examples. Intuitively, the {\em update} should become the bottleneck for eager approach since the materialized views should be refreshed whenever updates happen while the {\em All members} read should be the major overhead for lazy approach since the content of the virtual views should be computed whenever read queries are issued, which are the major concerns for the authors of Hazy. In terms of {\em update} queries, the authors assume that the time to retrain the model is negligible (roughly on the order of 100$\mu$s on their datasets) and the major overhead for view maintenance is to relabel the entities in the materialized views for eager approach. 

\subsubsection{Overview of view maintenance in Hazy (eager approach)}
%overview of view maintenance problem
In what follows, the details on how to maintain materialized classification views in Hazy are provided, which targets at more efficient {\em update} queries for eager approach rather than recomputing the content of views from the scratch when updates happen. Lazy approach is free from this issue since updates won't influence the virtual views in the database. 

%introduce notation for table H
Every time when updating the views is defined as an {\em epoch}. Suppose support vector machine is used thereafter (other classifiers are also applicable and introduced later) for binary classifications (extensions for multiple classifications are introduced later), at epoch $i$, the model parameters will be $(\bar{w^{(i)}}, b^{(i)})$. Given an entity with feature vector $\bar{f}$, its label is determined by the sign of the residual $eps = \bar{w^{(i)}}\cdot\bar{f}-b^{(i)}$. If $eps > 0$ ($eps < 0$ respectively), then this entity is labeled as $+$ ($-$ respectively).

Recall the assumption that the time to retrain the model is not significant, the model is firstly retrained at each epoch. Given the updated model parameters, a scratch table $\textbf{H}$ is maintained inside Hazy, which stores tuples of the form $(id, \bar{f}, eps)$ where $id$, $\bar{f}$ and $eps$ represents the entity id, feature vector and residual (computed by $eps = \bar{w^{(i)}}*\bar{f}-b^{(i)}$). Then the classification view $\textbf{V}$ can be obtained from $\textbf{H}$, in which each view tuple has the form $(id, c, eps)$ ($c$ is the label for entity with identifier $id$ and is calculated as $c=sign(eps)$). 
%For a tuple $t \in \textbf{H}$ ($\in \textbf{V}$ resp.), we use $t.attr$ denote the value of attribute $attr$ at tuple $t$ where $attr$ is an attribute of $\textbf{H}$ ($\textbf{V}$ respectively). For example, $t.\bar{f}$ and $t.id$ represent the feature vector and the entity at tuple $t$ in $\textbf{H}$.

%introduce two steps
\paragraph{Incremental step VS reorganization step} At epoch $i$, the scratch table $\textbf{H}$ is represented as $\textbf{H}^{(s)}$ which indicates that $\textbf{H}$ is constructed at epoch $s$ ($s < i$) and keeps the same after epoch $s$ until epoch $i$, view $\textbf{V}^{(i+1)}$ can be either {\em incrementally} constructed from $\textbf{V}^{(i)}$ by updating small portion of view tuples in $\textbf{V}^{(i)}$ without changing $\textbf{H}^{(s)}$ or computed by referencing latest $\textbf{H}$ which is {\em reorganized} at epoch $i$ (i.e. $\textbf{H}^{(s)}$ becomes $\textbf{H}^{(i)}$). Intuitively, the {\em incremental} step is relatively cheaper since it only touches parts of the view content and does not modify $\textbf{H}$. However, compared to the model constructed at epoch $s$, which is the latest {\em reorganization step}, more and more tuples in $\textbf{V}$ need to be updated since current model may be far away from the model at epoch $s$ as the time goes by. So the {\em reorganization step} is needed at some point to reduce the overhead to update the labels in $\textbf{V}$ in the following {\em incremental steps}.

%cost of the two options
\paragraph{Cost measure and skiing strategy} Due to the existence of two alternative choices at each epoch, i.e. {\em incremental step} and {\em reorganization step}, we should determine whether {\em incremental step} or {\em reorganization step} is taken at each epoch such that the overall overhead is minimized. A sequence of the choice of {\em reorganization step} or {\em incremental step} at each epoch is defined as a {\em strategy}. In order to determine the optimal strategy, Hazy quantifies the cost for the {\em incremental step} and {\em reorganization step} and formalizes it as the classic {\em ski rental problem} \cite{karlin1994competitive}. In this problem, the cost of the {\em incremental step} is defined as the time to update $\textbf{V}^{(i)}$ to $\textbf{V}^{(i+1)}$, which should be proportional to the view tuples relabeled at epoch $i$ (denoted as $c^{(i)}$) while the {\em reorganization step} has a fixed cost $S$. Suppose the {\em reorganization step} is taken at epoch $s$, then the accumulated cost at epoch $i$ by applying {\em incremental step} since epoch $s$ is $a^{(i)} = \sum_{j=s+1}^ic^{(j)}$. The strategy ({\em skiing strategy}) is as follows: If the accumulated cost $a^{(i)}$ is too large, say, greater than $\alpha S$ ($\alpha$ is a constant), then refresh table $H$ (the accumulated cost $a^{(i)}$ is reset to 0 after that). This strategy is proven to be 2-approximation asymptotically compared to the optimal strategy.


\subsubsection{Details of incremental step}\label{Sec: incremental step}
%overview
The core idea of {\em incremental step} is to do small changes over the classification view to reflect the model updates, the details of which are presented as below. It starts by introducing how to determine which part of view tuples to be changed at each epoch with two thresholds, which follows by the detailed derivation rules of the two thresholds.

\paragraph{Thresholds in incremental step}
%intro to lower water and higher water
For incremental step, it is assumed that the appearance of the new training examples do not result in great changes to the model, which means that it is safe to simply update small portion of the view content to achieve performance gains at each epoch. Intuitively, the labels of the entities that are far away from (close to resp.) the decision boundary should be less likely (more likely resp.) to be flipped when the updates happen. Its effect is presented in Figure \ref{fig:Hazy_view_update}, in which the model targets at classifying database papers. The small variation over the model, i.e. $\delta_w$ over $w$, has no influence on the labels of the samples far away from the decision boundary such as entity $P_4$.

\begin{figure}
    \centering
    \includegraphics[height = 0.3\textwidth, width=0.6\textwidth]{Figures/Hazy_view_update.png}
    \caption{The effect of small updates on the model over the views}
    \label{fig:Hazy_view_update}
\end{figure}


To determine which samples should be relabeled or not at each epoch, Hazy introduces two thresholds, lower water and higher water (denoted by $lw$ and $hw$ respectively, which are negative and positive respectively). At each epoch $i$, by referencing the residual value $eps$ for each entity in the scratch table $\textbf{H}^{(s)}$ computed at epoch $s$ ($s<i$), if $eps$ is greater than $hw$ (less than $lw$), then the labels of the corresponding entities are guaranteed to be $+$ ($-$ respectively) and thus remain unchanged. The process is exemplified in Figure \ref{fig:Hazy_threshold}, in which the entities to be reclassified are highlighted.


\begin{figure}
    \centering
    \includegraphics[height = 0.3\textwidth, width=0.6\textwidth]{Figures/Hazy_threshold.png}
    \caption{Determining the entities to be relabeled in the view in incremental step}
    \label{fig:Hazy_threshold}
\end{figure}

%how to compute lower water and higher water
\paragraph{How to determine the thresholds}
Intuitively, the two thresholds $hw$ and $lw$ should be adjusted at each epoch $i$ since the differences between model at epoch $i$ and model at epoch $s$ (when $\textbf{H}$ is recomputed) may be varied for different $i$. So $hw$ and $lw$ are associated with superscript $^{(s,i)}$ to indicate that they are related to the models at epoch $s$ and $i$, i.e. $hw^{(s,i)}$ and $lw^{(s,i)}$ respectively, which are computed as follows:

\begin{equation}
    lw^{(s,i)} = min_{l=s,\dots, i}\epsilon_{low}^{(s,l)},
    hw^{(s,i)} = max_{l=s,\dots, i}\epsilon_{high}^{(s,l)}
\end{equation}

where $\epsilon_{low}^{(s,l)}$ and $\epsilon_{high}^{(s,l)}$ are computed as (recall that $t.\bar{f}$ represents the feature vector of $t$):
\begin{equation}\label{eq: epsilon}
    \epsilon_{low}^{(s,l)} = -max_{t \in \textbf{H}}||t.f||_q||\bar{w^{(s)}}-\bar{w^{(l)}}||_p+b^{(l)}-b^{(s)}, \epsilon_{high}^{(s,l)} = max_{t \in \textbf{H}}||t.f||_q||\bar{w^{(s)}}-\bar{w^{(l)}}||_p+b^{(l)}-b^{(s)}
\end{equation}
where the norm $q$ and $p$ satisfy $p^{-1} + q^{-1} = 1$. Recall that $\bar{w^{(s)}}$, $b^{(s)}$ and $\bar{w^{(i)}}$, $b^{(i)}$ represents the model parameters at epoch $s$ and $i$ respectively.


There is a nice property by applying Holder's inequality \cite{rudin1976principles} to $\epsilon_{low}^{(s,j)}$ and $\epsilon_{low}^{(s,j)}$, i.e. at epoch $j$, for a tuple $t \in \textbf{H}^{(s)}$, if $t.eps \geq \epsilon_{high}^{(s,j)}$ ($t.eps \leq \epsilon_{low}^{(s,j)}$), then $t.id$ should be in the class $+$ ($-$). The proof is sketched as follows:

% Define $\delta \bar{w} = \bar{w}^{(j)}-\bar{w}^{(s)}$. 
\begin{proof}
To prove that $t.id$ is in class $+$ at epoch $l$, we need to prove that $t.f\cdot \bar{w}^{(l)} - b^{(l)} > 0$. According to Equation \ref{eq: epsilon},
$\epsilon_{high}^{(s,l)} = max_{t \in \textbf{H}}||t.f||_q||\bar{w^{(s)}}-\bar{w^{(l)}}||_p+b^{(l)}-b^{(s)} \geq ||t.f||_q||\bar{w^{(s)}}-\bar{w^{(l)}}||_p+b^{(l)}-b^{(s)}$. By applying Holder's inequality, we have $||t.f||_q||\bar{w^{(s)}}-\bar{w^{(l)}}||_p \geq ||t.f||||\bar{w^{(s)}}-\bar{w^{(l)}}|| \geq ||t.f \cdot(\bar{w^{(s)}}-\bar{w^{(l)}})||$. $\epsilon_{high}^{(s,l)} = max_{t \in \textbf{H}}||t.f||_q||\bar{w^{(s)}}-\bar{w^{(l)}}||_p+b^{(l)}-b^{(s)} \geq ||t.f||_q||\bar{w^{(s)}}-\bar{w^{(l)}}||_p+b^{(l)}-b^{(s)} \geq ||t.f \cdot (\bar{w^{(s)}}-\bar{w^{(l)}})|| + b^{(l)}-b^{(s)} \geq t.f \cdot (\bar{w^{(s)}}-\bar{w^{(l)}}) + b^{(l)}-b^{(s)}$. Note that $t.eps = t.f \cdot \bar{w}^{(s)} - b^{(s)}$ If $t.eps \geq \epsilon_{high}^{(s,l)}$, then $t.eps \geq t.f \cdot (\bar{w^{(s)}}-\bar{w^{(l)}}) + b^{(l)}-b^{(s)} = t.eps - t.f \cdot \bar{w}^{(l)} + b^{(l)}$, which means that $t.f\cdot \bar{w}^{(l)} - b^{(l)} > 0$.
\end{proof}

\subsubsection{Details of reorganization step}
The {\em reorganization step} for the scratch table $\textbf{H}$ includes recomputing the residual $\epsilon$ for each entity in $\textbf{H}$, reconstructing necessary indexes to quickly search the entity in $\textbf{H}$ and sorting the entire table by the residual $eps$.\eat{, which happens when the current model is far away from the model computed in the last reorganization step. A greedy strategy is proposed earlier to determine whether {\em reorganization step} or {\em incremental step} is chosen at each epoch.  Formally speaking, when the cost of {\em reorganization step} (fixed as $S$, which is the time to perform the {\em reorganization step}) is less than the accumulated cost $a_{i}$, {\em reorganization step} is executed.}

\subsubsection{Analysis of the greedy strategy}
Two assumptions about the cost measure are made for {\em skiing strategy}: 1) the cost of incremental step, $c^{(s, i)}$ only depends on the current epoch $i$ and the most recent epoch $s$ for reorganization step since $c^{(s, i)}$ is a function of the number of tuples within the interval $[lw^{(s,i)}, hw^{(s,i)}]$; 2) Reorganization more recently won't increase the cost $c^{s,i}$, which means that $c^{(s, i)} < c^{(s',i)}$ for $s > s'$. The assumptions above can guarantee 2-approximation compared to theoretical optimal strategy. The analysis is sketched as follows.


%notations
Recall that the total overhead for reorganization step is $S$, in which the time to scan $\textbf{H}$ is $\sigma S$. Given $N$ epochs in total, reorganization step is executed $M$ times, which happens at epoch $\mu_1, \mu_2,\dots, \mu_M$ ($0 \leq \mu_1 < \mu_2 < \dots < \mu_M \leq N$), The integer sequence $\bar{\mu} = \mu_1, \mu_2,\dots, \mu_M$is defined as a {\em schedule} under a certain strategy. At any epoch $i$, we define $\floor{i}_{\bar{\mu}} = max\{\mu \in \bar{\mu} | \mu < i\}$, which intuitively returns the most recent epoch before epoch $i$ when {\em reorganization step} is executed. Denote the set of costs for every epoch by $\bar{c} = \{c^{(s,i)}\}$, the total cost of a {\em schedule} will be:
\begin{equation}\label{eq: cost}
    Cost(\bar{\mu}, S, \bar{c}) = \sum_{i=1,2,\dots,N}c^{(\floor{i}_{\bar{\mu}}, i)} + MS
\end{equation}

Equation \ref{eq: cost} indicates that the total cost not only depends on how to schedule the reorganization step, but also depends on the cost at each epoch, i.e., $\bar{c}$ and $S$. This is because at each epoch, the value of $c^{(\floor{i}_{\bar{\mu}}, i)}$ is known to us only after the model is updated and thus $\epsilon^{(s,l)}_{low}$ and $\epsilon^{(s,l)}_{high}$ in Equation \ref{eq: epsilon} can be derived. (recall that $c^{(s,i)}$ is related to the interval $[lw^{(s,i)}, hw^{(s,i)}]$ and $lw^{(s,i)}$, $hw^{(s,i)}$ are derived from $\epsilon^{(s,l)}_{low}$ and $\epsilon^{(s,l)}_{high}$), which can influence the accumulated cost $a^{(i)}$ and thus influence the decision (recall that the decision at epoch $i$ depends on whether $a^{(i)}$ is greater than $\alpha S$). So given a strategy $\Phi$, the schedule $\bar{\mu}$ should be a function of $\bar{c}$, i.e. $\bar{\mu} = \Phi(\bar{c})$.

In order to prove that the skiing strategy is a 2-approximation strategy, the {\em competitive ratio} (denoted by $\rho$) is defined as below, which is a ratio between the cost of a strategy $\Phi$ and and the optimal cost:
\begin{equation}
    \rho(\Phi) = sup_{\bar{c}}\frac{Cost(\bar{\mu}, S, \bar{c})}{Cost(\bar{o}, S, \bar{c})}
\end{equation}


where $\bar{o}$ represents the optimal schedule. It is proven that $\rho(skiing) = (1+\alpha + \sigma)$ where $\alpha$ is the positive root of $x^2 + \sigma x - 1$ and for any other deterministic strategy, $\rho(\Phi) \geq (1+\alpha + \sigma)$. Specifically, when the number of entities goes to infinity, the time to scan $\textbf{H}$ table, i.e. $\sigma S$ should be approaching 0 (recall that $S$ is the reorganization time, which includes the time to sort $\textbf{H}$. The sorting time is more expensive than scanning time). In this case, $\alpha \rightarrow 1$ and thus $\rho(skiing) \rightarrow 2$.

% The proof of $\rho(skiing) = (1+\alpha + \sigma)$ is sketched as follows:
% 
% \begin{proof}
% 1) first we prove that $\rho(skiing) = (1+\alpha + \sigma)$. 

% Suppose the {\em reorganization step} happens at epoch $t_1, t_2, \dots, t_M$. Consider any interval $[t_m, t_{m+2})$, in which the cost in the interval $(t_m, t_{m+1})$ and $(t_{m+1}, t_{m+2})$ is $C_1$ and $C_2$ respectively. So the total cost in this interval would be $C_1 + C_2 + S$. Since the reorganization step happens at epoch $t_{m+1}$ and $t_{m+2}$ respectively, it means that $C_i \geq \alpha S$ ($i=1,2$) according to the skiing strategy. Besides, at epoch $t_{m+1}$, 


% \end{proof}

\subsubsection{Overview of read queries in Hazy (lazy approach)}
%overview
In this section, how Hazy improves the performance on read queries, i.e. {\em single entity} read and {\em all members} read is presented. Recall that in eager approach, the classification views are materialized. So read queries are not a bottleneck here since the labels of either single entity or all the entities can be retrieved from the view instance directly. On contrast, only the virtual views are used in lazy approach, which may slow down the read queries since the view content should be generated whenever read queries are issued. As a result, the authors focus on improving the performance of read queries (especially {\em all members} queries) for lazy approach.

Suppose a user wants to retrieve all the entities of the positive class, which is a typical {\em all members} query, in the naive solution, at a certain epoch $i$, all the entities need to be scanned to be labeled with the current model. However, in Hazy, not all the entities are needed to answer this query. Instead, by referencing the scratch table $\textbf{H}^{(s)}$ (reorganized at epoch $s$, $s < i$), only the tuples $t \in \textbf{H}$ satisfying $t.eps > lw^{(s,i)}$ are needed (recall that entities with residual $t.eps < lw^{(s,i)}$ are guaranteed to be in the negative class). 

Same as view maintenance problem introduced in the previous few sections, for eager approach, there also exists a trade-off on whether the scratch table $\textbf{H}$ should be reorganized. For the read queries under the lazy approach, the greedy strategy is applied to determine when to do reorganization but with different cost measure to quantify the overhead of {\em incremental step}. Suppose it takes $S'$ seconds to read the entire entities, the number of real positive entities is $N_{+}$ and the number of entities above the low water $lw^{(s,i)}$ is $N_{R}$, then the cost of {\em all members} queries will be $c^{(i)} = \frac{N_{R}-N_{+}}{N_{R}}S'$ and the accumulated cost is calculated as in Section \ref{Sec: incremental step}. Then the rest of the skiing strategy here is identical to the one used for eager approach.


\subsubsection{Other optimizations in Hazy}
Hazy also provides two types of system variants to achieve the performance enhancement. The first one is to maintain the classification views and the scratch table $\textbf{H}$ in memory (refered as Hazy-MM), which can be safely discarded when reorganizations happen since the model parameters will be recomputed at each epoch. The other one is to hybrid the main-memory structure and the on-disk structure to reduce the memory consumption. The performance trade-offs between the original system design and the two variants are explored experimentally.

\subsubsection{Extends to other machine learning models}
In the previous sections, it is assumed that the default machine learning algorithm is support vector machine and the problems Hazy is dealing with belong to binary classification problem. But Hazy is also flexible to handle other machine learning models (with kernel methods) and multi-class classification.

\paragraph{Other machine learning model}
Other linear classification models are also supported by Hazy, which includes ridge regression and logistic regression. Actually, typical linear classification model can be formalized as {\em convex optimization problem} to derive the best parameter $\bar{w}$, which has same form as Equation \ref{eq: logistic_objective_function} and is usually solved with gradient descent methods. Then the label of an entity with feature values $\bar{x}$ is determined as:

\begin{equation}
    l(\bar{x}) = h(\bar{w}\cdot\bar{x})
\end{equation}
% \begin{equation}\label{eq: opt_expr}
%     min_{\bar{w}}P(\bar{w}) + \Sigma_{(\bar{x},y) \in \textbf{T}}L(\bar{w}\cdot\bar{x}, y)
% \end{equation}

% where $\bar{x}$ is the feature vector, $P$ is the {\em regularization term} and $L$ is the {\em loss function}. Typical solution to optimize the objective function shown in Equation \ref{eq: opt_expr} is to apply gradient descent methods to derive $\bar{w}$ iteratively until it converges to $\bar{w}*$, which is then used to classify a certain sample with feature values $\bar{x}$ as follows:

where $h$ is simply sign function for binary classification problem.

Since the objective function of other linear classification models share the same form as support vector machine, it is pretty straightforward to migrate the solutions to those classification models.

\paragraph{Kernel methods}
Kernel methods extend linear classification models with kernel function $K: \mathbb{R}^d \times \mathbb{R}^d \rightarrow \mathbb{R}$, which is a positive semi-definite function. Given a kernel function, the corresponding classification function would be:
\begin{equation}\label{eq: kernel classification}
    l(\bar{x})=h(\Sigma_{i=1,2,\dots,N}c_i\cdot K(\bar{s_i}, \bar{x}))
\end{equation}
where $\bar{s}_i$ is the {\em support vector} and $c_i$ is the weight value for the $i_{th}$ kernel function. In this case, the weight vector $\bar{w}$ is: $\bar{w} = [c_1, c_2,\dots, c_N]$, which can still fit the eager approach and lazy approach proposed before.

%linearized kernels
Some further improvements can be applied to a special type of kernel functions called shift-invariant kernels, which satisfy $K(x, y) = K(x-y)$ where $K$ is a kernel function and regarded as a simple function. Shift-invariant kernels includes many common kernels, such as the Gaussian and the Laplacian kernel. By applying {\em random non-linear feature vectors} \cite{rahimi2008random}, $K(x, y) \approx z(x)^Tz(y)$ where $z$ is a random map: $\mathbb{S}^d \rightarrow R^D$ (suppose $x, y \in \mathbb{S}^d$), which can thus simplify Equation \ref{eq: kernel classification} as below:
\begin{equation}
    l(\bar{x})=h(\Sigma_{i=1,2,\dots,N}c_i\cots K(\bar{s_i}, \bar{x})) = h(\Sigma_{i=1,2,\dots,N}c_i\cots z(\bar{s_i})^T z(\bar{x}))=h(\bar{v}^T z(\bar{x}))
\end{equation}

where $v=\sum_{i=1,\dots,N}c_i z(\bar{s_i})$. This technique can substantially reduce the dimension.

\paragraph{Multi-class classification}
Hazy builds a decision-tree-like structure to turn a multi-classification problem to binary classification problem.

\subsubsection{Experimental results}
The authors also conducted experiments on three different datasets, i.e. Forest, DBlife and Citeseer to compare the time performance between the skiing strategy (Hazy) and Naive strategy under both on-disk and in memory architecture, which are presented in Table \ref{tab:hazy_exp}. In those experiments, the average time per update in Eager approach and the average time per {\em All members} read query in Lazy approach are recorded.

As Table \ref{tab:hazy_exp} shows, in most cases, Hazy achieves at least one order-of-magnitude speed-up relative to Naive implementation even in on-disk architecture. But there is still one exception on update queries over Citeseer datasets, in which Hazy fails to outperform Citeseer. The reason behind that is that Citeseer dataset has larger feature space and thus the model fails to converge by the end of each epoch, which forces Hazy to conduct more unnecessary reorganization steps and thus pay more cost on sorting operation in those steps. But in-memory architecture incurs much less overhead in sorting and thus Hazy is more efficient than Naive implementation.


\begin{table}[h]
    \centering
    \begin{tabular}{|c|c|c|c|c|c|c|c|}\hline
        && \multicolumn{3}{|c|}{Eager {\em Update} (updates/s)}&\multicolumn{3}{|c|}{Lazy {\em All members} (scan/s)}  \\\hhline{~~------}
        % &&Re-evaluation & Incremental\\ \hline
        &&Forest&DBLife & Citeseer & Forest & DBLife & Citeseer\\\hline
    \multirow{2}{*}{On disk}&Naive&0.4 & 2.1 & 0.2 & 1.2 & 12.2 & 0.5\\\hhline{~-------}
    &Hazy&2.0 & 6.8 & 0.2 & 3.5 & 46.9 & 2.0\\\hline
    % \multicolumn{2}{|c|}{Hybrid}&2.0 & 6.6 & 0.2 & 8.0 & 48.8 & 2.1\\\hline
    \multirow{2}{*}{In memory}&Naive&5.3 & 33.1 & 1.8 & 10.4 & 65.7 & 2.4\\\hhline{~-------}
    &Hazy&49.7 & 160.5 & 7.2 & 410.1 & 2.8k & 105.7\\\hline
    % \multirow{3}{*}{Space}&Linear & $n^2$ & $n^2k$\\\hhline{~---}
    % &Exponential & $n^2$ & $n^2log k$\\\hhline{~---}
    % &Skip-s & $n^2$ & $n^2(logs+\frac{k}{s})$\\\hline
    \end{tabular}
    \caption{Experiemental results in Hazy}
    \label{tab:hazy_exp}
\end{table}


% \subsection{Discussion}

\subsection{Incrementally maintain the output of general linear algebra programs}
In data analytics, linear algebra programs are quite ubiquitous, which usually involve updates to the input matrix data. To effectively propagate the updates through the linear algebra program to the output, there are some efforts which materialize the output of the linear algebra programs as views and incrementally maintain the view content. One example is Linview \cite{nikolic2014linview}, which expresses the updates to the input as {\em delta expression} and targets at propagating such {\em delta expression} through the iterative linear algebra programs with basic matrix operations, i.e. matrix addition, multiplication, subtraction, transpose and inverse, to update the result of the programs rather than re-evaluate it from the scratch. In what follows, how the authors of \cite{nikolic2014linview} model the iterative linear algebra programs, represent and propagate {\em delta expressions} through the linear algebra program and deal with typical linear algebra programs are elaborated respectively.

\subsubsection{Model the iterative programs}\label{sec: iterative_model}
Iterative computation is very common in linear algebra programs, for which naive re-evaluation from the scratch in the case of data updates can incur high overhead. Under the assumption that the iterative programs end after a fixed number of iteration steps (even after updates), Linview deals with delta expression propagation through the program under three different iterative models.

\paragraph{Linear model} The first iterative model is {\em linear model}, which simply updates the result at $k_{th}$ epoch based on the results at $(k-1)_{th}$ epoch. Given the input $A$, the output at epoch $k$, $T^{(k)}$ is computed as:

\[T^{(k)}=
\begin{cases}
f(A)& k=1\\
g(T^{(k-1)}, A) & k=2,3,\dots\\
\end{cases}
\]

where $f$ and $g$ represent the modified version of the original linear algebra programs that fit the linear model.

\paragraph{Exponential model} In the {\em exponential model}, the output at $k_{th}$ epoch depends on the result at $(k/2)_{th}$ epoch, which has larger leaps compared to {\em linear model}, i.e.:

\[
T^{(k)}=
\begin{cases}
f(A)& k=1\\
g(T^{(k/2)}, A) & k=2,4,8\dots\\
\end{cases}
\]

\paragraph{Skip model} {\em Skip model} lies in between, in which the step between the computed iterations is adjustable. Given a skip size $s$, the result before $s_{th}$ epoch is computed by {\em exponential model} and then the result after $s_{th}$ epoch is computed every $s_{th}$ iteration. The {\em skip model} is thus represented as:

\[
T^{(k)}=
\begin{cases}
f(A)& k=1\\
g(T^{(k/2)}, A) & k=2,4,8\dots,s\\
h(T^{(k-s)}, T^{(s)}, A) & k=s, 2s,\dots
\end{cases}
\]

\subsubsection{Incremental computation}
This subsection is centered around how Linview represent and propagate delta expression through the linear algebra programs, which starts by introducing the incremental updates over matrix manipulation primitives. Matrix multiplications are frequently used in the following analysis, which is assumed to have time complexity $O(n^{\gamma})$ ($2\leq \gamma \leq 3$, varies for different algorithms)

\paragraph{Delta rules for basic matrix operations}
Given an $n \times n$ matrix $A$ and a function $f(*)$, small changes $\Delta A$ to $A$ leads to the changes of output of $f(A)$ as: $\Delta_A(f) = f(A+\Delta A) - f(A)$. By following the properties of basic matrix operations, such as matrix addition, matrix multiplication and matrix inverse, the delta rule is shown as follows:

\begin{center}
    \begin{minipage}{0.4\textwidth}
      \begin{itemize}
        \item $\Delta_A(f_1 f_2) = f_2\Delta_A(f_1) + f_1\Delta_A(f_2)$
        \item $\Delta_A(f_1 \pm f_2) = \Delta_A(f_1) \pm \Delta_A(f_2)$
        \item $\Delta_A(\lambda f) = \lambda \Delta_A(f)$
        \item $\Delta_A(f^{-1}) = f(A+\Delta A)^{-1}-f(A)^{-1}$
      \end{itemize}
    \end{minipage}
  \end{center}

Note that for most update rules shown above, the time complexity is $O(n^2)$ while in general case, the delta derivation for matrix inverse has the same overhead as reevaluation from the scratch, which is much more expensive than $O(n^2)$. However, under the assumption that $\Delta A$ is relatively smaller than $A$ and thus have lower rank ($k$) than the rank of $A$ (at most $n$) where $k \ll n$, $\Delta A$ can be decomposed into additions and multiplications of vectors and thus much cheaper for computation. The decomposition of $\Delta A$ is called {\em factored form}. For example, given a rank-1 $\Delta A$ ($=\bar{u}\bar{v}^T$), by applying Sherman-Morrison formula \cite{press2007numerical}, $\Delta_A(f^{-1})$ can be written as:

\begin{equation}\label{eq: matrix_inverse}
\Delta_A(f^{-1}) = -\frac{f(A)^{-1}\bar{u}\bar{v}^Tf(A)^{-1}}{1+\bar{v}^Tf(A)^{-1}\bar{u}}
\end{equation}

\paragraph{Delta representation}
The overhead of incremental updates also relies on how to represent the delta. It has turned out that naive representations won't help even for very simple program with very minor changes. 

\begin{example}\label{eg: incremental_naive_update}
For example, consider the following program to compute $A^8$ for an $n\times n$ matrix $A$:

\begin{center}
    $B:= AA$\\
    $C:= BB$\\
    $D:= CC$
\end{center}

Given a small change $\Delta A$, such as an update to the entry $A_{i,j}$, $\Delta A_{i,j}$, the time complexity of which is $O(1)$, by applying the delta rules for matrix multiplication mentioned above, $\Delta B$, $\Delta C$ and $\Delta D$ are represented as:

\begin{equation}\label{eq: delta_b}
    \Delta B = (\Delta A) A + A (\Delta A) + (\Delta A) (\Delta A)
\end{equation}
\begin{equation}\label{eq: delta_c}
    \Delta C = (\Delta B) B + B (\Delta B) + (\Delta B) (\Delta B)
\end{equation}
\begin{equation}\label{eq: delta_d}
    \Delta D = (\Delta C) C + C (\Delta C) + (\Delta C) (\Delta C)
\end{equation}
    


Computing $\Delta B$ only requires $O(n)$ operations since $(\Delta A) A$ and $A (\Delta A)$ only need to scale the $i_{th}$ row and $j_{th}$ column of $A$ by $\Delta A_{i,j}$ respectively, which takes $O(n)$ time while $(\Delta A) (\Delta A)$ only incurs $O(1)$ overhead. In the end, compared to original $B$, the changes caused by $\Delta B$ appear in the $i_{th}$ row and $j_{th}$ column of $B$.

Similarly, we can derive that $\Delta C$ requires $O(n^2)$ operations, which ends up with the propagation of the initial updates to the entire $C$. The effect of update propagation is shown in Figure \ref{fig:update_propagete}.

\begin{figure}
    \centering
    \includegraphics[width=10cm, height=4cm]{Figures/update_propagation.png}
    \caption{The effect of updates propagate}
    \label{fig:update_propagete}
\end{figure}

Then computing $\Delta D$ involves at least two full $O(n^{\gamma})$ matrix multiplications, which is obviously more expensive than recomputing $\Delta D$ from the scratch. \qed
\end{example}


To deal with such unexpected effect, the delta expressions are all represented as Example \ref{eg: incremental_naive_update} indicates, the updates $\Delta A$ is only a change to a single entry $A_{i,j}$, which is a rank-1 update and can be represented as $\Delta A = \bar{u}\bar{v}^T$ where $\bar{u}$ and $\bar{v}$ are both column vectors. Then by replacing $\Delta A$ in Equation \ref{eq: delta_b} with $\bar{u}\bar{v}^T$, $\Delta B$ is rewritten as:
\begin{equation}\label{eq: update_b_product}
\Delta B = \bar{u}(\bar{v}^TA) + (A\bar{u})\bar{v}^T + (\bar{u}\bar{v}^T\bar{u})\bar{v}^T=[\bar{u}, (A\bar{u}), (\bar{u}\bar{v}^T\bar{u})]
\begin{bmatrix}
    (\bar{v}^TA)  \\
    \bar{v}^T  \\
    \bar{v}^T   
\end{bmatrix}
=[\bar{u}_1, \bar{u}_2, \bar{u}_3]
\begin{bmatrix}
    \bar{v}^T_1  \\
    \bar{v}^T_2  \\
    \bar{v}^T_3  
\end{bmatrix}
\end{equation}

which is a sum of three outer products between vectors. The term $(\bar{v}^TA)$, $(A\bar{u})$ and $(\bar{u}\bar{v}^T\bar{u})$ can be all derived in $O(n^2)$ time while the vector $[\bar{u}, (A\bar{u}), (\bar{u}\bar{v}^T\bar{u})]$ and $[(A^T\bar{v}), \bar{v}, \bar{v}]$ are both $n \times 3$ matrices and thus computing $\Delta B$ in this way requires $O(3n^2)$ operations.

Although it is more expensive than naive updates for $\Delta B$ as shown in Example \ref{eg: incremental_naive_update}, in general, all the delta expressions can be expressed as a product of two $n \times k$ matrices where $k \ll n$, which only requires $O(kn^2)$ operations. For example, for $\Delta C$, since it is a sum of three products, i.e. $\Delta B B$, $B \Delta B$ and $\Delta B \Delta B$, and each of them can be expressed as a product of two $n \times 3$ matrices by inserting Equation \ref{eq: update_b_product} to Equation \ref{eq: delta_c}, $\Delta C$ can thus be expressed as a a product of two $n \times 9$ matrices:


\begin{equation}
\begin{split}
    \Delta C &= (\Delta B) B + B (\Delta B) + (\Delta B) (\Delta B) \\&=[\bar{u}_1, \bar{u}_2, \bar{u}_3]
\begin{bmatrix}
    \bar{v}^T_1B \\
    \bar{v}^T_2B \\
    \bar{v}^T_3B 
\end{bmatrix}+[B\bar{u}_1, B\bar{u}_2, B\bar{u}_3]
\begin{bmatrix}
    \bar{v}^T_1 \\
    \bar{v}^T_2 \\
    \bar{v}^T_3 
\end{bmatrix}\\
&+[\bar{u}_1, \bar{u}_2, \bar{u}_3]
\begin{bmatrix}
    \bar{v}^T_1  \\
    \bar{v}^T_2  \\
    \bar{v}^T_3  
\end{bmatrix}
[\bar{u}_1, \bar{u}_2, \bar{u}_3]
\begin{bmatrix}
    \bar{v}^T_1  \\
    \bar{v}^T_2  \\
    \bar{v}^T_3  
\end{bmatrix}\\
&=[\bar{u}_1, \bar{u}_2, \bar{u}_3, B\bar{u}_1, B\bar{u}_2, B\bar{u}_3, \bar{u}_1, \bar{u}_2, \bar{u}_3]\\
&\begin{bmatrix}
    \bar{v}^T_1B \\
    \bar{v}^T_2B \\
    \bar{v}^T_3B \\
    \bar{v}^T_1  \\
    \bar{v}^T_2  \\
    \bar{v}^T_3  \\
    \bar{v}^T_1(\bar{u}_1\bar{v}^T_1 + \bar{u}_2\bar{v}^T_2 + \bar{u}_3\bar{v}^T_3)\\
    \bar{v}^T_2(\bar{u}_1\bar{v}^T_1 + \bar{u}_2\bar{v}^T_2 + \bar{u}_3\bar{v}^T_3)
    \bar{v}^T_3(\bar{u}_1\bar{v}^T_1 + \bar{u}_2\bar{v}^T_2 + \bar{u}_3\bar{v}^T_3)
\end{bmatrix}
\end{split}
\end{equation}

which requires $O(9n^2)$ operations in total. Finally, by inserting the expression of $\Delta C$ above into Equation \ref{eq: delta_d}, computing $D$ only requires $O(27n^2)$ operations, which is far cheaper than re-evaluation from the scratch.

Actually, $\Delta B$ can be further optimized as:
\begin{equation}\label{eq: update_b_product_opt}
\Delta B = \bar{u}(\bar{v}^TA) + (A\bar{u})\bar{v}^T + (\bar{u}\bar{v}^T\bar{u})\bar{v}^T=[\bar{u}, (A\bar{u}) + (\bar{u}\bar{v}^T\bar{u})]
\begin{bmatrix}
    (\bar{v}^TA)  \\
    \bar{v}^T 
\end{bmatrix}
=[\bar{u}_1', \bar{u}_2']
\begin{bmatrix}
    \bar{v}^T_1'  \\
    \bar{v}^T_2'  \\
\end{bmatrix}
\end{equation}

which is a product of two $n \times 2$ matrices. Consequently, $\Delta C$ and $\Delta D$ can be represented as a product of two $n \times 4$ matrices and two $n \times 8$ matrices respectively.

\paragraph{In the case of programs with multiple inputs}
Given the derivation rules above, Linview can also handle the case where the linear algebra programs have multiple inputs and each of them may involve updates. \eat{converts the linear algebra program into a set of trigger functions, which takes a set of input matrices and handle the updates of them sequentially. }For example, given a set of input matrices $D=\{A, B, \dots\}$ and an evaluation function $f(*)$ over that, then the overall effect of updates will to $f(D)$ will be:

\begin{equation}\label{eq: updates_by_multi_vars}
\Delta_{D}(f(D)) = \Delta_{A}(f(D)) + \Delta_{D \setminus \{A\}}(f(D) + \Delta_{A}(f(D)))
\end{equation}


%  and Linview generates one trigger function for each of them, which is then used to monitor the change of the corresponding input matrix.

\subsubsection{Incremental analysis for typical programs}
The incremental update model above can fit some typical linear algebra programs, such as ordinary least squares, matrix powers and even more general forms, which are illustrated below.

\paragraph{Regression model}
Recall that in Section \ref{sec: pre}, for regression model, given an $m \times n$ matrix $X$ for predictor variables and an $m \times p$ matrix $Y$ for response variables,   the coefficient $W^*$ is estimated as $W^* = (X^TX)^{-1}X^TY$.

In practice, the regression model is usually built in a dynamic environment where the incoming new data requires efficient incremental updates to $W^*$ instead of computing from the scratch every time, for which the performance bottleneck is to update the inverse operation $(X^TX)^{-1}$. We represent $X^TX$ as $Z$. So given a rank-1 update to $X$, i.e. $\Delta X = \bar{u}\bar{v}^T$, $\Delta Z$ can be expressed as:
\begin{equation}\label{eq: regression_factor}
    \Delta Z = [\bar{v}\ \ (X^T\bar{u} + \bar{v}\bar{u}^T\bar{u})]\begin{bmatrix}
    \bar{u}^T X  \\
    \bar{v}^T  
\end{bmatrix}\\
=[\bar{p}_1\ \bar{p}_2]\begin{bmatrix}
    \bar{q}^T_1  \\
    \bar{q}^T_2  
\end{bmatrix}=\bar{p}_1\bar{q}_1^T + \bar{p}_2\bar{q}_2^T
\end{equation}

The time complexity of Equation \ref{eq: regression_factor} is $O(mn)$ due to the computation of $X^T\bar{u}$, $\bar{v}\bar{u}^T\bar{u}$ and $\bar{u}^TX$, which ends up with a product of two $n\times 2$ matrices and thus a rank-2 update to $Z$. We can further represent $\Delta Z$ as the sum of $\Delta Z_1$ and $\Delta Z_1$ where $\Delta Z_1 = \bar{p}_1\bar{q}_1^T$ and $\Delta Z_2 = \bar{p}_2\bar{q}_2^T$ such that $\Delta Z_1$ and $\Delta Z_2$ are both rank-1 matrix. Considering the fact that both $X$ and $\Delta X$ are $m \times n$ matrices and thus $\bar{u}$ and $\bar{v}$ are $m \times 1$ and $n \times 1$ vector respectively, $\bar{p}_1$ and $\bar{q}_2$ should be both $n \times 1$ vectors. Besides, the computation result of $X^T\bar{u}$ and $\bar{v}\bar{u}^T\bar{u}$ are both $n \times 1$ vector and thus $\bar{p}_2$ (sum of $X^T\bar{u}$ and $\bar{v}\bar{u}^T\bar{u}$) should be also $n \times 1$ vector. Similarly, $\bar{q_1}^T$ is also $n \times 1$ vector.

By following Equation \ref{eq: matrix_inverse} and Equation \ref{eq: updates_by_multi_vars}, the update rules for $J = (X^TX)^{-1} = Z^{-1}$ with respect to $\Delta Z_1$ and $\Delta Z_2$ will be:

\begin{center}
\begin{equation}
\begin{split}
    \Delta_{Z_1}(J)& = -\frac{J\bar{p}_1\bar{q}_1^TJ}{1 + \bar{q}_1^TJ\bar{p}_1}\\
    \Delta_{Z_2}(J)& = -\frac{(J+\Delta_{Z_1}(J))\bar{p}_2\bar{q}_2^T(J+\Delta_{Z_1}(J))}{1 + \bar{q}_2^T(J+\Delta_{Z_1}(J))\bar{p}_2}
\end{split}   
\end{equation}
\end{center}

while simply computes the rank-1 update $\Delta Z_1$ to $J$ and then accumulate the effect when propagating $\Delta Z_2$ for $J$. $\Delta_{Z_1}$ and $\Delta_{Z_2}$ can be further written as:

\begin{equation}
\begin{split}
\Delta_{Z_1}(J) = \bar{r}_1\bar{s}_1^T\\    
\Delta_{Z_2}(J) = \bar{r}_2\bar{s}_2^T
\end{split}
\end{equation}

where $\bar{r}_1 = J\bar{p}_1$, $\bar{r}_2 = (J+\Delta_{Z_1}(J))\bar{p}_2$ while $\bar{s}_1$ and $\bar{s}_2$ are the rest of the subexpressions of $\Delta_{Z_1}(J)$ and $\Delta_{Z_2}(J)$ respectively. So the overall updates to $J$ is $\Delta J = \bar{r}_1\bar{s}_1^T + \bar{r}_2\bar{s}_2^T$, which is a rank-2 update.
Since $\bar{p}_i$ and $\bar{q}_i$ ($i=1,2$) are column vector of length $n$ and $Z$ and $J$ are $n \times n$ matrices, the computation of $\bar{r}_1 = W\bar{p}_1$ thus need $O(n^2)$ operations, which is the same for $\bar{s}_1$, $\bar{s}_2$ and $\bar{r}_2$. So the overall cost to update $W$ is thus $O(n^2 + mn)$, which is far less than the cost of recomputing from the scratch, i.e. $O(n^3 + mn)$.

In terms of the update for $W^*$ (denoted by $\Delta W$), given the update to the input $X$, i.e. $\Delta X=\bar{u}\bar{v}^T$, $\Delta W$ can be thus written as:

\begin{equation}
    \Delta W = \Delta JX^TY + J\Delta X^TY + \Delta J\Delta X^TY = (\bar{r}_1\bar{s}_1^T + \bar{r}_2\bar{s}_2^T)X^TY + J\bar{v}\bar{u}^TY + (\bar{r}_1\bar{s}_1^T + \bar{r}_2\bar{s}_2^T)\bar{v}\bar{u}^TY
\end{equation}

while requires $O(n^2 + mp + np + mn)$ operations in total while re-evaluation incurs $O(mnp + n^2min(m,p))$ operations.

\paragraph{Matrix powers}
The goal of matrix powers is to compute $A^k$ for an input matrix $A$ with a given exponent $k$, which plays an important role in many data analysis tasks. Matrix powers can be computed iteratively, for which the three iterative models proposed in \ref{sec: iterative_model} are applicable and the trade-offs between the three models are explored under re-evaluation strategy or incremental update strategy are presented in Figure \ref{fig:time_space_complexity_matrix_power} under the assumption that $k \ll n$.

% \begin{figure}
%     \centering
%     \includegraphics[width=8cm, height=4.5cm]{Figures/time_space_complexity_matrix_powers.png}
%     \caption{The time and space trade-offs for matrix power}
%     \label{fig:time_space_complexity_matrix_power}
% \end{figure}
\begin{table}[h]
    \centering
    \begin{tabular}{|c|c|c|c|}\hline
        &\multirow{2}{*}{Model} & \multicolumn{2}{|c|}{(Sum of) Matrix Powers}  \\\hhline{~~--}
        &&Re-evaluation & Incremental\\ \hline
    \multirow{3}{*}{Time}&Linear & $n^{\gamma}k$ & $n^2k^2$\\\hhline{~---}
    &Exponential & $n^{\gamma}logk$ & $n^2k$\\\hhline{~---}
    &Skip-s & $n^{\gamma}(logs+\frac{k}{s})$ & $n^2\frac{k^2}{s}$\\\hline
    \multirow{3}{*}{Space}&Linear & $n^2$ & $n^2k$\\\hhline{~---}
    &Exponential & $n^2$ & $n^2logk$\\\hhline{~---}
    &Skip-s & $n^2$ & $n^2(logs+\frac{k}{s})$\\\hline
    \end{tabular}
    \caption{The time and space trade-offs for (sum of) matrix power}
    \label{tab:time_space_complexity_matrix_power}
\end{table}

For re-evaluation strategy, it would require $O(n^{\gamma})$ operations for matrix multiplication at every iteration and thus the total time complexity depends on the number of iterations. Thus Exponential model outperforms other models since it has largest leap between iterations. But all of the models have the same space complexity, i.e. $O(n^2)$.

For incremental update strategy, the computation time at each iteration should be $O(rn^2)$ where $r$ is the rank of the input delta expression on the updates at each iteration and the rank of delta expression increases linearly as the iteration proceed, which leads to the nearly quadratic running time and linear space consumption with respect to iteration number.

Similar to {\em matrix power}, the sum of matrix power problem, which aims at computing $S^{(k)} = I + A + A^2 + \dots + A^k$ given input $A$ and maximum exponent $k$ where $I$ is the identity matrix, have the almost the same computation process and thus have the same time complexity as shown in Table \ref{tab:time_space_complexity_matrix_power}. 

\paragraph{General form} The authors also consider a general iterative matrix computation programs based on matrix powers, i.e. $T^{(i+1)} = AT^{(i)} + B$ where $A$ is the input matrix with dimension $n \times n$, $B$ is a constant matrix with dimension $n \times p$ and the result $T^{(i)}$ is iteratively computed which hhas dimension $n \times p$. Such general form appears in many applications such as PageRank and power iteration method for eigenvalue computation. 

Note that $T^{(i+1)} = AT^{(i)} + B$ can be unrolled as the form for computing $T^{(i+k)}$ with the results at iteration $i$ and previous iterations, i.e. $T^{(i+k)} = A^kT^{(i)} + (A^{k-1} + A^{k-2} + \dots + A + I) B$, in which how to efficiently incrementally compute $P^{(k)} = A^k$ and $S^{(k)} = (A^{k-1} + A^{k-2} + \dots + A + I)$ has been discussed before and thus both $P^{(k)}$ and $S^{(k)}$ are materialized and maintained along with $T^{(k)}$. Given those notations, the derivation rules for $T^{(i+1)} = AT^{(i)} + B$ under the three iterative models from Section \ref{sec: iterative_model} are shown in Table \ref{tab:derivation_rule}.

\begin{table}[]
    \centering
    \begin{tabular}{|c|c|}\hline
        Model & Derivation rule \\ \hline
        Linear & $T^{(k)}=
\begin{cases}
AT^{(0)} + B& k=1\\
AT^{(i-1)} + B & k=2,3,4\dots,k
\end{cases}$\\ \hline
        Exponential & $T^{(k)}=
\begin{cases}
AT^{(0)} + B& k=1\\
P^{(i/2)}T^{(i/2)} + S^{(i/2)}B & k=2,4,8\dots,s\\
\end{cases}$\\ \hline
        Skip-s &$T^{(k)}=
\begin{cases}
AT_0 + B& k=1\\
P^{(i/2)}T^{(i/2)} + S^{(i/2)}B & k=2,4,8\dots,s\\
P^{(s)}T^{(i-s)} + S^{(s)}B & k = 2s, 3s, \dots, k
\end{cases}$\\ \hline
    \end{tabular}
    \caption{Derivation rules for $T^{(i+1)} = AT^{(i)} + B$ under three iterative models}
    \label{tab:derivation_rule}
\end{table}

Then given an update $\Delta A$ against the input matrix $A$, the cost analysis under different iterative models and update strategies is provided below.

In terms of re-evaluation strategy, the results of $T^{(i)}$, $P^{(i)}$ and $S^{(i)}$ are only materialized for current iteration. For linear model, computing $AT^{(i)}$ is the performance bottleneck, which requires $O(np)$ multiplications  (the result has dimension $n \times p$) and thus $O(pn^2)$ time in total at each operation. So the total time complexity is $O(kpn^2)$ for $k$ iterations. For exponential model and skip-s model, maintaining $P^{(i)}$ and $S^{(i)}$ would incur time complexity $O(n^{\gamma})$ for every iterations, which is essential for $logk$ and $logs$ iterations respectively. Besides the cost of maintaining the auxiliary matrices $P^{(i)}$ and $S^{(i)}$, it still needs $O(pn^2)$ time to compute $P^{(s)}T^{(i-s)}$ or $P^{(i/2)}T^{(i/2)}$ at each iteration. Since the number of iterations for exponential model and skip-s model is $logk$ and $logs + \frac{k}{s}$ respectively, the overall time complexity for the two models will be $O(n^{\gamma} + pn^2)logk$ and $O(n^{\gamma}logs + pn^2(logs+ \frac{k}{s}))$.

For incremental strategy, the time complexity to incrementally update $P^{(s)}$ and $S^{(s)}$ is still $O(rn^2)$ for each iteration as discussed before where $r$ is the rank of the factored forms of $P^{(s)}$ and $S^{(s)}$. To derive $T^{(k)}$, the update rule $P^{(i/2)}T^{(i/2)} + S^{(i/2)}B$ or $P^{(s)}T^{(i-s)} + S^{(s)}B$ is used, which requires extra $O(np)$ time to compute the factored form for $T^{(i)}$. So the overall time complexity for each iteration will be $O(n^2 + np)$. By following similar analysis for {\em matrix powers}, the time complexity and space complexity for the three different iterative models is presented in Table \ref{tab:time_space_complexity_general_form}.

But notice that for some extreme case, such as $p=1$, incremental update strategy has worse performance than re-evaluation strategy, which is due to the unnecessary factor form representations for $T^{(i)}$ (a vector in this case). To solve this problem, a combination of the re-evaluation strategy and incremental updates strategy is proposed, which simply represent delta expression for $T^{(i)}$ as a single matrix instead of two factored vectors and still represent the auxiliary matrix $P^{(i)}$ and $S^{(i)}$ in factored form. So for exponential model and skip-s model, they require $rn^2$ operations to update $P^{(i)}$ and $S^{(i)}$ for each iteration where $r$ is the rank of them, which incurs the same overhead as the incremental updates for matrix powers while the time complexity to compute $T^{(k)}$ is $O(pn^2)$ when $P^{(i)}$ and $S^{(i)}$ are given, which has the same time complexity as re-evaluation strategy. By summing up the time complexity from the two parts, the overall time complexity is presented in Table \ref{tab:time_space_complexity_general_form}. Since both $P^{(i)}$ and $S^{(i)}$ for hybrid strategy are materialized in the same form as incremental update strategy for every iteration, it thus shares the same space complexity as incremental update strategy.


% \begin{figure}
%     \centering
%     \includegraphics[width=12cm, height=4.5cm]{Figures/time_space_complexity_general_form.png}
%     \caption{The time and space trade-offs for general forms}
%     \label{fig:time_space_complexity_general_form}
% \end{figure}

\begin{table}[h]
    \centering
    \begin{tabular}{|c|c|c|c|c|}\hline
        &\multicolumn{4}{|c|}{$T^{(i+1)} = AT^{(i)} + B$}  \\\hhline{~----}
        &Model&Re-evaluation & Incremental&Hybrid\\ \hline
    \multirow{3}{*}{Time}&Linear & $pn^2k$ & $(n^2+pn)k^2$&$pn^2k$\\\hhline{~----}
    &Exponential & $(n^{\gamma}+pn^2)logk$ & $(n^2+pn)k$&$pn^2logk+n^2k$\\\hhline{~----}
    &Skip-s & $n^{\gamma}logs+pn^2(logs +  \frac{k}{s})$ & $(n^2+pn)\frac{k^2}{s}$&$pn^2(logs + \frac{k}{s}) + n^2s$\\\hline
    \multirow{3}{*}{Space}&Linear & $n^2 + np$ & $n^2+knp$&$n^2 + knp$\\\hhline{~----}
    &Exponential & $n^2 + np$ & $(n^2 + np)logk$& $(n^2 + np)logk$\\\hhline{~----}
    &Skip-s & $n^2 + np$ & $(n^2+np)logs+np\frac{k}{s})$&$(n^2+np)logs+np\frac{k}{s})$\\\hline
    \end{tabular}
    \caption{The time and space trade-offs for general forms}
    \label{tab:time_space_complexity_general_form}
\end{table}


\subsection{Discussions}
In this section, some recent work on incrementally updating views in the context of data analysis tasks is summarized, where the views can be model parameters \cite{deshpande2006mauvedb, gupta2015processing}, classification results \cite{koc2011incrementally} and the output of general linear algebra programs \cite{nikolic2014linview}. 

\paragraph{Other related work} The authors of \cite{deshpande2006mauvedb} also developed another system called FunctionDB \cite{thiagarajan2008querying}, which supports query over continuous function in relational database in the context of sensor network, in which the continuous functions are either derived by applying regression model over discrete data or from the continuous data directly. Unlike \cite{deshpande2006mauvedb} where the data or the functions are gridded and materialized as views as the response to user queries, \cite{thiagarajan2008querying} provide symbolic representations for query plans for continuous function and grid the answers in the end, which guarantees more accurate and less noisy answers.

Actually, the machine learning model-based view maintenance idea from \cite{deshpande2006mauvedb} and \cite{gupta2015processing} falls under a larger topic on model management and machine learning life-cycle \cite{crankshawmissing} where machine learning models are trained to provide service for users on line and their interactions with users can produce new data which are necessary to update the models with low latency, which are achieved in a distributed system, Velox.

In \cite{koc2011incrementally}, the authors explored an efficient way to update classification results on model updates, which involves residual computation and relabeling at each epoch when updates come. A follow-up work of this is DeepDive \cite{shin2015incremental}, which is a system targeting at extracting structured information from a collection of unstructured documents to construct knowledge base (KCB) with some weighted rules. The knowledge base construction involves two phases, i.e. {\em grounding phase} and {\em inference phase}, which are executed alternatively until convergence. To capture the iterative characteristic of the KCB process, {\em incremental inference} is come up with to incrementally update the rule weights.

As a follow-up work for \cite{nikolic2014linview}, \cite{nikolic2018incremental} generalizes the same {\em factorization} idea to other data analysis applications where incremental view maintenance also emerge but with different specifications for multiplications and additions for semiring compared to \cite{nikolic2014linview}. For example, in SQL query, the multiplication and addition are specified as join and aggregation operation respectively. By leveraging similar {\em factored form}s from \cite{nikolic2014linview}, the incremental view maintenance tasks achieve order of magnitude speed-up.

\paragraph{Limitations} In terms of \cite{deshpande2006mauvedb} and \cite{gupta2015processing}, they are dealing with almost the same problem with similar strategies, which, however, still have some drawbacks. First of all, the machine learning models that \cite{deshpande2006mauvedb} and \cite{gupta2015processing} can handle are quite limited, which only include very simple linear models, such as linear regression, logistic regression and Naive Bayes. Whether it is possible to migrate their solutions to other complicated machine learning models like neural networks is still agnostic and suspicious. Besides, even though for those simple models, there are still some performance-wise questions. For example, recall that for linear regression, both \cite{deshpande2006mauvedb} and \cite{gupta2015processing} only materialize the matrix product results $H^TH$ and $H^T\bar{y}$ in Equation \ref{eq: regression_solve}, which are incrementally modified for the incoming updates and then used to compute the model parameter $\bar{w}^*$ by invoking Equation \ref{eq: regression_solve_final}. However, a matrix inversion operation exists in Equation \ref{eq: regression_solve_final}, i.e. $(H^TH)^{-1}$, in which is $H^TH$ is a $k\times k$ matrix and $k$ represents the number of features. Although \cite{deshpande2006mauvedb} claims that $k$ is small in general, there are many applications where the dataset is of high-dimensions and thus the feature number is far more than the number of data points, e.g. genomic data analysis \cite{buhlmann2011statistics}, in which the matrix inversion becomes the major overhead such that the incremental maintenance strategies proposed in \cite{deshpande2006mauvedb} and \cite{gupta2015processing} for linear regression won't help improve the performance compared to re-evaluation from the scratch. Last but not least, note that the updates can be either addition or removal of data points used for machine learning model construction. However, for most machine learning algorithms dealt by \cite{deshpande2006mauvedb} and \cite{gupta2015processing} except linear regression and Naive Bayes, the updates including the removal of data points are not supported. It is thus worth thinking about enabling provenance support \cite{cheney2009provenance} for those machine learning models such that the effect of the deletion of certain data points (used for machine learning training process) can be reflected in the result.

\cite{koc2011incrementally} provides an efficient greedy algorithm on how to update the classification results stored in RDBMS, which relies on one assumption that the time to retrain the model parameters to reflect the updates is negligible. However, the assumption is not satisfiable, especially for some complicated machine learning models. 


We also observe that \cite{nikolic2014linview} is a kind of generalizations to \cite{deshpande2006mauvedb, gupta2015processing} since deriving model parameters are usually computed by some linear algebra programs. For example, to derive the coefficient $\bar{w}^*$ for linear regression, Equation \ref{eq: regression_solve_final} is used, which can be regarded as a linear algebra program composed of matrix multiplication and inversion. So to incrementally update $\bar{w}^*$ in Equation \ref{eq: regression_solve_final}, the strategies proposed in \cite{nikolic2014linview} are applicable and can avoid expensive matrix inversion operations in the case of low-rank updates, which are inevitable in \cite{deshpande2006mauvedb} and \cite{gupta2015processing} as mentioned before. However, despite its potential for general linear algebra programs, some obvious limitations of \cite{nikolic2014linview} prevent its widely use in general data analysis tasks since its delta representations for low-rank updates only support basic matrix operations such as matrix multiplications, additions and inversions, the usability of which is questionable in some complex operators or functions over matrices appearing in typical machine learning algorithms, e.g. the logistic function in logistic regression.
\section{Incrementally derive new models with existing models}
In the last section, we reviewed related works on how to maintain views in the context of data analysis tasks, where the views can be the model parameters, classification results and the results of linear algebra programs. However, given those predefined {\em views}, how to make those views usable for further data analysis tasks becomes another challenge, which is similar to traditional query rewriting using views problem where proper views are selected for answering queries \cite{halevy2001answering}.

In this section, \cite{hasani2018efficient} and part of \cite{gupta2015processing} are introduced, which borrow the ideas from query optimization to construct approximate machine learning models by model {\em reusing} the {\em materialized} model. The high-level idea is that the user issues a ``query'' on some data point in a dataset for building a machine learning model, which is then approximately (but efficiently) constructed by reusing and combining a set of ``relevant'' pre-materialized models. When building the model to answer users' requests, \cite{hasani2018efficient} proposes two different approaches to combining relevant pre-materialized models, i.e., {\em merging} the pre-materialized model parameters or combining their {\em coresets}, which incur different overhead and theoretical guarantee in terms of approximation. \eat{ but can deal with very general types of machine learning models, such as generalized linear models (GLMs), K-means and Guassian Mixture Models (GMMs).} On the other hand, only the first approach in employed in \cite{gupta2015processing}. Similar to query optimization problem, there might be multiple ways to derive the model for user's request, which requires the one with minimal cost.

Considering the similarity between \cite{hasani2018efficient} and \cite{gupta2015processing}, in what follows, some basic concepts from \cite{hasani2018efficient} are introduced, which follow by the two different approaches to combining the pre-materialized models (as mentioned before) along with an algorithm on how to find the minimal-cost strategy for some limited type of queries and then follow by the sketched solutions to general queries. In the end, the differences between \cite{hasani2018efficient} and \cite{gupta2015processing} are highlighted.


\subsection{Basic concepts}
The dataset that \cite{hasani2018efficient} is dealing with is a relation $\textbf{D}$ with $n$ tuples, $d$ attributes $\bar{\textbf{a}} = \{\textbf{a}_1, \dots, \textbf{a}_d\}$, which may have hierarchical structures (e.g. attribute $city$ with hierarchical structure $City \rightarrow State \rightarrow Country$) and can be divided into feature attributes $\bar{\textbf{x}}$ and labels attributes $\bar{\textbf{y}}$.

We can issue some queries to extract some tuples from $\textbf{D}$ via some predicates, such as {\em range based predicates} (e.g. $\textbf{a}_1 \in [lb, ub]$) and {\em dimension based predicates} (e.g. $\textbf{a}_2 = `c'$) over some attributes, which are then used for training a machine learning model. The entire approach has two phases, i.e. ``pre-computing phase'' and ``running time phase''. During ``pre-computing phase'', a set of pre-materialized machine learning models $\mathcal{M} = \{M_1, M_2, \dots, M_n\}$ are prepared for answering user query and each $M_i$ corresponds to a query predicate over $\textbf{D}$ which is used to extract tuples from $D$ to construct $M_i$. In the following ``running time phase'', given a user query with a predicate, a set of candidate pre-materialized models are selected for approximately building the requested model.

\begin{example}\label{eg: two_phase_eg}
For example, given a relation $\textbf{D}=\{1,2,\dots, 1000\}$, in the pre-computing phase, the pre-materialized models $\{M_0, M_1, M_2, M_3, M_4\}$ can be built over the tuples from $\textbf{D}$ specified by the range predicate $P_0 = [100, 300], P_1 = [250, 500], P_2 = [500, 1000], P_3 = [300, 900], P_4 = [900, 1000]$. 

In the running time phase, a user query with predicate $q=[250, 1000]$ is submitted, for which we can combine $M_1$ and $M_2$ to compute the model for $q$.
\end{example}

As mentioned before, \cite{hasani2018efficient} targets at very general machine learning algorithms, such as K-means, GMMs and GLMs, the notations of which are briefly presented below:

\paragraph{K-means} K-means is an unsupervised clustering algorithm, which aims at computing a set of centroids $C$ from a set of data points $X$ and assigning each data point from $X$ to one of centroid in $C$ by minimizing the sum of square errors (SSE) with a similarity measure $d(*)$ in it:

\begin{equation}\label{eq: sse_k_means}
    SSE(X, C) = \sum_{\bar{x} \in X}d(\bar{x}, C)
\end{equation}

\paragraph{GMM} GMM is the probabilistic version of K-means, which is parameterized by a set of model parameters $\bar{\theta} = \{(w_1, \bar{\mu_1}, \Sigma_1), (w_2, \bar{\mu_2}, \Sigma_2), \dots, (w_k, \bar{\mu_k}, \Sigma_k)\}$ such that the probability of a data point $\bar{x}$ belonging to a cluster $i$ is computed by the probability density function of normal distribution and weighted by $w_i$, i.e.  $w_i\frac{exp(-\frac{1}{2}(\bar{x}-\bar{\mu}_i)^T\Sigma^{-1}(\bar{x}-\bar{\mu}_i))}{\sqrt{(2\pi)^k|\Sigma|}}$. The model parameters are derived during the training process with EM algorithm.

\paragraph{GLM} GLM includes a large class of typical linear classifiers such as logistic regression and support vector machine (SVM) and also typical regression methods, such as linear regression.

\subsection{Constructing models for user query}
In this subsection, assuming that there exists a set of pre-materialized models $\mathcal{M} = \{M_1, M_2, \dots, M_r\}$ built on different portions of $D$ and there is a user query $q$ over the subset of $D$, i.e. $D_q$, two approaches (i.e. {\em merging model} and {\em coreset construction}) used to obtain an approximate model $\tilde{M}_q$ for $q$ are presented below. 

\subsubsection{Constructing by merging models}
Suppose we can find a set of candidate pre-materialized machine learning models $\mathcal{M}_q (\subseteq \mathcal{M})$ usable for a user query $q$, {\em merging model} approach simply merges the parameters from each model in $\mathcal{M}_q$ to compute the parameters of the approximate model $\tilde{M}_q$, which varied across different machine learning algorithms.

\paragraph{Merging model for K-means}
For K-means, each pre-materialized model will store the centroids as the parameters. Suppose the union of the centroids from $\mathcal{M}_q$ is $C_w$ and the cluster represented by a centroid $c_j$ in $C_w$ includes $w_j$ data points from $\textbf{D}$. Then K-means++ \cite{arthur2007k} runs over $C_w$ with weight $w_j$ for each centroid $c_j \in C_w$ to produce k centroids as output, which will be a $O(log k)$-approximate model compared to the one constructed on $D_q$ directly and a $O(log^2k)$-approximate model compared to the optimial model on $D_q$.

\paragraph{Merging model for GMM}
The main takeaway of the solutions to K-means above is to clustering centroids from the candidate models with further invocations of the clustering algorithm, which, does not work for GMM, although GMM is a generalized model for K-means. This is because 1) the model parameters of GMM include mean vector, covariance matrix and a prior probability, which are far more complicated than K-means and far more expensive to estimate; 2) the merged model may be far away from the one built from the scratch. 

To overcome the issues above, the authors proposed an iterative approach to merge the GMM clusters from the candidate models $\mathcal{M}_q$, which starts by unioning all the cluster parameters from $\mathcal{M}_q$ (denoted by $\theta_q$) with the cluster parameters in the form of $(w_i, \bar{\mu}_i, \Sigma_i)$.
Then the most similar model pairs with parameters $(w_1, \bar{\mu}_1, \Sigma_1)$ and $(w_2, \bar{\mu}_2, \Sigma_2)$ are determined by Bhattacharyya distance \cite{bhattacharyya1943measure} (denoted by $D_B(*)$, see Equation \ref{eq: db_distance}), which are merged into a single cluster by following Equation \ref{eq: GMM_merging}.

\begin{equation}\label{eq: db_distance}
\begin{split}
    D_B(\bar{\mu}_1, \bar{\mu}_2, \Sigma_1, \Sigma_2) &=
    \frac{1}{8}(\bar{\mu}_1-\bar{\mu}_2)^T\Sigma^{-1}(\bar{\mu}_1 -\bar{\mu}_2) \\&+ \frac{1}{2}ln(\frac{|\Sigma|}{\sqrt{|\Sigma_1||\Sigma_2|}})\\
    \Sigma &= \frac{\Sigma_1 + \Sigma_2}{2}
\end{split}
\end{equation}

\begin{equation}\label{eq: GMM_merging}
    \begin{split}
        w &= w_1 + w_2\\
        \bar{\mu} &= \frac{1}{w}(w_1\bar{\mu}_1 + w_2\bar{\mu}_2)\\
        \Sigma &= \frac{w_1}{w}[\Sigma_1 + (\bar{\mu}_1 - \bar{\mu})^T(\bar{\mu}_1 -\bar{\mu})]\\
        &+\frac{w_2}{w}[\Sigma_2 + (\bar{\mu}_2 - \bar{\mu})^T(\bar{\mu}_2 -\bar{\mu})]
    \end{split}
\end{equation}

The merging process continues until there are exact k clusters in the end.


\paragraph{Merging model for classifiers}
Suppose for each model $M_i \in \mathcal{M}_q$, the model parameter is $\theta(M_i)$, in order to derive the approximate model $\tilde{M}_q$ for $q$, the model parameters from $\mathcal{M}_q$ are simply averaged, i.e. $\theta(\tilda{M}_q) = \frac{1}{|\mathcal{M}_q|}\Sigma_{M \in \mathcal{M}_q}\theta(M)$, which, however, has been proven by \cite{zhang2012communication} to match the error rate of the model built from the scratch on $D_q$.

\paragraph{Merging model in \cite{gupta2015processing}} Recall that \cite{gupta1993maintaining} deals with different machine learning model from \cite{hasani2018efficient}, i.e. linear regression, Naive Bayes and logistic regression. Logistic regression belongs to GLM and thus the solution mentioned before for classifiers fit very well for it, which is thus not discussed here. But the authors of \cite{hasani2018efficient} show that merging linear models or Naive Bayes models can end up with ``exact'' updated model parameters, which is presented below.

Recall that in Section \ref{sec: view_maintenance_model}, given a set of training data points $D = \{\bar{x}_i, y_i\}(i=1,2,\dots,n)$, the model parameters to be maintained for linear regression model are $H^TH$ and $H^Tf$, where in many cases, $H$ and $f$ are:

\begin{equation}
    H=\begin{bmatrix}
x_{11} & x_{12} &\dots &x_{1k}\\
x_{21} & x_{22} &\dots &x_{2k}\\
\dots\\
x_{n1} & x_{n2} &\dots &x_{nk}\\
\end{bmatrix}
=\begin{bmatrix}
\bar{x}_1\\
\bar{x}_2\\
\dots\\
\bar{x}_n\\
\end{bmatrix}
\end{equation}


\begin{equation}
    \bar{f} = \{y_1, y_2,\dots, y_n\}^T
\end{equation}

So $H^TH$ and $H^f$ have the following form:
\begin{equation}
    H^TH=\begin{bmatrix}
HH_{1,1} & HH_{1,2} &\dots HH_{1,k}\\
HH_{2,1} & HH_{2,2} &\dots HH_{2,k}\\
\dots\\
HH_{n,1} & HH_{n,2} &\dots HH_{n,k}\\
\end{bmatrix}=\begin{bmatrix}
\Sigma_{j=1}^nx_{j1}x_{j1} & \Sigma_{j=1}^nx_{j1}x_{j2} &\dots \Sigma_{j=1}^nx_{j1}x_{jk}\\
\Sigma_{j=1}^nx_{j2}x_{j1} & \Sigma_{j=1}^nx_{j2}x_{j2} &\dots \Sigma_{j=1}^nx_{j2}x_{jk}\\
\dots\\
\Sigma_{j=1}^nx_{jk}x_{j1} & \Sigma_{j=1}^nx_{jk}x_{j2} &\dots \Sigma_{j=1}^nx_{jk}x_{jk}\\
\end{bmatrix}
\end{equation}

\begin{equation}
    H^Tf=\begin{bmatrix}
Hf_{1,1} & Hf_{1,2} &\dots Hf_{1,k}\\
Hf_{2,1} & Hf_{2,2} &\dots Hf_{2,k}\\
\dots\\
Hf_{n,1} & Hf_{n,2} &\dots Hf_{n,k}\\
\end{bmatrix}=\begin{bmatrix}
\Sigma_{j=1}^nx_{j1}y_{j}\\
\Sigma_{j=1}^nx_{j2}y_{j}\\
\dots\\
\Sigma_{j=1}^nx_{jk}y_{j}\\
\end{bmatrix}
\end{equation}

Given two models with materialized parameters $H_1^TH_1$, $H_1^Tf_1$ and $H_2^TH_2$, $H_2^Tf_2$ respectively which are constructed over two sets of data points $D_1$ and $D_2$, if $D_1 \bigcap D_2 = \Phi$, then they are merged as follows (suppose the materialized paramters for the merging model are $H^TH$, $H^Tf$ respectively):

\begin{equation}
\begin{split}
{HH}_{i,j} &= {H_1H_1}_{i,j} + {H_2H_2}_{i,j}\\
{Hf}_{i} &= {H_1f_1}_{i} + {H_2f_2}_{i}
\end{split}
\end{equation}

Otherwise,

\begin{equation}
\begin{split}
{HH}_{i,j} &= {H_1H_1}_{i,j} + {H_2H_2}_{i,j} - \Sigma_{\bar{x}_k \in D_1 \bigcap D_2}x_{ki}x_{kj} \\
{Hf}_{i} &= {H_1f_1}_{i} + {H_2f_2}_{i} - \Sigma_{\bar{x}_k \in D_1 \bigcap D_2}x_{ki}y_{k} 
\end{split}
\end{equation}

Similarly, for Naive Bayes model, since the values of $N_c$, $S_{jc}$ and $SS_{jc}$ are materialized for maintenance by referencing Section \ref{sec: view_maintenance_model}, given two models with materialized values $N1_c$, $S1_{jc}$, $SS1_{jc}$ and $N2_c$, $S2_{jc}$, $SS2_{jc}$ which are constructed over two sets of data points, $D_1$ and $D_2$, the materialized parameters (denoted by $N_c$, $S_{jc}$ and $SS_{jc}$) for the merging model are:

\begin{equation}
\begin{split}
N_c &= N1_c + N2_c-N_c'\\
S_{jc} &= S1_{jc} + S2_{jc}-S_{jc}'\\
SS_{jc} &= SS1_{jc} + SS2_{jc}-S_{jc}'\\
\end{split}
\end{equation}

where $N_c'$, $S_{jc}'$ and $SS_{jc}'$ are materialized parameters for the model built on the data points in $D_1 \bigcap D_2$.

\subsection{Answering query by coreset}
Merging model approach is very efficient since it constructs the approximate model for the user query without having to access the data points but it lacks theoretical guarantee compared to the model built from the scratch for some machine learning algorithms, which can be alleviated by coreset approach.

Specifically, a weighted set of data points $C \subseteq \textbf{D}$ is a $\epsilon-$coreset for $D \subseteq \textbf{D}$ if $C \subseteq D$ and $(1-\epsilon)\phi(D) \leq \phi(C) \leq (1+\epsilon)\phi(D)$ where $\phi$ is the loss function (e.g. Equation \ref{eq: sse_k_means}) and $\phi(D)$ and $\phi(C)$ are the evaluation of $\phi$ over $D$ and $C$. Intuitively, the coreset is a subset of $D$, $C$ such that the machine learning model built on (weighted) $C$ and $D$ are close enough.

Recall that there are two computation phases for constructing approximate model for user query, i.e. {\em pre-computing phase} and {\em running time phase}. In {\em pre-computing phase}, the coreset is constructed for every pre-materialized model while in {\em running time phase}, the coreset is used to construct approximate model for user query, which are illustrated below.

\paragraph{Coreset in pre-computing phase} The coreset is constructed during {\em pre-computing phase} by one common strategy, i.e. sampling the data points proportional to their ``importance'' to the loss function $\phi$. The ``importance'' is quantified by a surrogate function which should be selected to 1) have good approximation ratio compared to the loss function $\phi$; 2) guarantee efficient computation; 3) be  agnostic about the optimal solution computed by $\phi$. So the surrogate function should be varied across different machine learning algorithms. For example, for K-means with loss function shown in Equation \ref{eq: sse_k_means}, given a set of data points $D_i$, the surrogate function is defined as follow:

\begin{equation}\label{eq: surrogate_function}
    p(x) = \frac{1}{2}\frac{1}{|D_i|} + \frac{1}{2}\frac{d(x, \mu(D_i))^2}{\Sigma_{x'\in D_i}d(x', \mu(D_i))^2}
\end{equation}

where $\mu(*)$ is used to compute the mean of all $D_i$. This equation is efficient since it only requires two passes over the entire $D_i$ to compute the importance for every $x \in D_i$. After computing the probability for every data point in $D_i$, we can do the probability sampling over the data points from $D_i$ to select $m$ data points where $m$ is the input sampling size, which guarantees that the result is $\epsilon-$coreset according to the proof in \cite{bachem2017scalable}.

\paragraph{Coreset in running time phase} There are two intriguing properties for coreset, which is beneficial to model construction in the running time phase. The first property is the {\em compositional} property, i.e. given the $\epsilon-$coresets $C_1$ and $C_2$ for two datasets $D_1$ and $D_2$, $C_1 \bigcup C_2$ should be also an $\epsilon-$coreset for $D_1 \bigcup D_2$, which can produce a coreset of large size for $D_q$ (the data points that the user query touches) if we union the coresets from all the candidate models from $\mathcal{M}_q$. 

To solve this issue, we employ the second property of coreset, i.e. the size of an $\epsilon-$coreset simply depends on $\epsilon$ and a probability value $\delta$ but independent from the dataset size. For example, with the probability $1-\delta$, to generate a $\epsilon-$coreset, which should have the size at least $\Omega(\frac{dk+log\frac{1}{\delta}}{\epsilon^2})$ where $k$ is the cluster number and $d$ is the dimension number. Suppose each model in $\mathcal{M}_q$ has a coreset of size $m$, by following this property, the coreset construction algorithm is applied over the union of all the coresets from models in $\mathcal{M}_q$ (a set of size $m|\mathcal{M}_q|$) to generate a coreset of size $m$ for $D_q$.


\subsection{Optimizing the execution strategy}\label{sec: opt}
In the previous subsection, we assume that the candidate model set $\mathcal{M}_q$ is given for a user query $q$ and analyze how to construct approximate models with $\mathcal{M}_q$ with {\em merging model} approach or {\em coreset} approach. However, in practice, there can be multiple possible options for $\mathcal{M}_q$. For example, let us revisit Example \ref{eg: two_phase_eg}, to construct model for query $q$, there are two candidate model sets. The first one is $\{M_1, M_2\}$, which exactly covers the range predicate of $q$ and thus can be combined directly to build the model for $q$. The other option is to train an auxiliary model $M'$ for the range $[250, 300]$ and combine $M'$, $M_3$ (with predicate $[300, 900]$) and $M_4$ (with predicate $[900, 1000]$) for the same purpose. 

\paragraph{Cost measure} To compare those different options and thus find the best ones, the authors introduce cost measures for the cost of building model from the scratch $C_{build}$, combining models with {\em merging model} approach (denoted by $C_{merge}$) or {\em coreset} approach (denoted by $C_{coreset}$) to evaluate the total overhead for each option. For Example \ref{eg: two_phase_eg}, suppose {\em merging model} approach is used, then the cost of the first option and second option above will be $C_{merge}(M_1) + C_{merge}(M_2)$ and $C_{build}(M') + C_{merge}(M_3) + C_{merge}(M_4)$ respectively. 

\paragraph{Building execution strategy graph} Based on the cost measure, the problem of determining the min-cost strategy is formulated as the shortest path problem in the {\em execution strategy graph}, where the node set is composed of the distinct lower bound and upper bound values of all the range predicates from the pre-materialized models as well as the query (with predicate $[lb, ub]$) while weighted edge is built from node $a$ to node $b$ such that $a < b$. The weight on every edge from node $a$ to $b$ represents the cost to ``cover'' the interval $[a,b]$, which depends on whether there exists a pre-materialized model built over this interval. If not, then the weight is $C_{build}([a,b])$. Otherwise, the weight will be $C_{merge}([a,b])$ for merging model approach or $C_{merge}([a,b]) + C_{coreset}([a,b])$ for coreset approach. Then we run Dijkstra's algorithm over the graph to determine the shortest path from the node $lb$ to the node $ub$. Note that before the graph constructing process, only the ``relevant'' pre-materialized models are considered, which have predicates fully included by the query predicate. Let us revisit Example \ref{eg: two_phase_eg} to see how the graph is constructed.

\begin{example}
In Example \ref{eg: two_phase_eg}, $M_0$ is not relevant model for $q$ since the corresponding range predicate $[100, 300]$ is not a subset of the query predicate $[250, 1000]$, which is thrown away. For the predicates of other models which are all relevant and the query predicates, all the lower bound and upper bound values compose the following set: $\{250, 300, 500, 900, 1000\}$, each of which becomes a node in the graph. Assuming the the merging model approach is used, the edges and corresponding weights are then constructed by following the rules mentioned above, which ends up with a graph in Figure \ref{fig:min_cost_graph}. Finally, Dijkstra's algorithm is applied over this graph to derive the shortest path from the node 250 to the node 1000 and thus determine the option with minimal cost.

\begin{figure}[t]
\begin{center}
\begin{tikzpicture}[grow'=up,scale=0.75]
% \tikzset{vertex/.style = {shape=circle,draw,minimum size=1.5em}}
% \tikzset{edge/.style = {->,> = latex'}}
% vertices
\node[vertex] (1) at (-2,0) {$250$};
\node[vertex] (2) at (6,0) {$300$};
\node[vertex] (3) at (-4,4) {$500$};
\node[vertex] (4) at (2,8) {$900$};
\node[vertex] (5) at (8,4) {$1000$};
%edges
% \draw[edge] (1) to (3);
% \path[->]
% (1) edge[bend left] node[swap] {$\neg$} (2)
% (2) edge[bend left] node {$\neg$} (1);
\scriptsize{
\draw[->] (1) --(3) node[pos=0.5,left]{$C_{merge}([250, 500])$}; 
\draw[->] (1) --(2) node[pos=0.5,below]{$C_{build}([250, 300])$}; 
\draw[->] (2) --(3) node[pos=0.6,right]{$C_{build}([300, 500])$}; 
\draw[->] (3) --(5) node[pos=0.5,above]{$C_{merge}([500, 1000])$};
\draw[->] (2) --(4) node[pos=0.15,above]{$C_{merge}([300, 900])$};
\draw[->] (4) --(5) node[pos=0.7,right]{$C_{merge}([900, 1000])$};
\draw[->] (3) --(4) node[pos=0.5,left]{$C_{build}([500, 900])$};
\draw[->] (1) --(5) node[pos=0.2,above]{$C_{build}([250, 1000])$};
\draw[->] (2) --(5) node[pos=0.5,right]{$C_{build}([300, 1000])$};
\draw[->] (1) --(4) node[pos=0.7,above]{$C_{build}([250, 900])$};}
% \draw[edge] (2) to (3); 
% \draw [->] (-0.4,0.3) arc (10:330:20pt);
\end{tikzpicture}
\end{center}
\caption{Graph to determining the option with minimal cost}
\label{fig:min_cost_graph}
\end{figure}

\end{example}

\paragraph{Building execution strategy graph in \cite{gupta2015processing}} Observe that when combining two or models to construct new model for the user query in \cite{hasani2018efficient}, more and more data points from $\textbf{D}$ are included rather than removed. For example, in Example \ref{eg: two_phase_eg}, we need to combine the model $M_3$ and $M_4$ toward the construction of the model for query $q$, $M_q$ and the data points that $M_3$ and $M_4$ are built on, i.e. the data points in $[300, 900]$ and $[900, 1000]$, are unioned to indicate that those data points are necessary to build $M_q$.  In contrast, during the combinations of pre-materialized models for linear regression and Naive Bayes, \cite{gupta2015processing} can also handle the case where the unnecessary data points are removed from those models, which thus result in different concepts of {\em relevant} pre-materialized models and thus different ways to construct the execution strategy graphs.

Given a query $q$ with a range predicate $P_q = [lb,ub]$, if a pre-materialized model $M_i$ with range predicate $P_i$ is relevant to $q$ if the intersection of $P_i$ and $P_q$ is not empty or the intersection of $P_i$ and the range predicate of some relevant pre-materialized models is not empty. 

\begin{example}\label{eg: relevant_models}
For example, if a query $q'$ has range predicate $[100, 500]$, for the pre-materialized models provided in Example \ref{eg: two_phase_eg}, the relevant models should include all of them. Note that although the range predicate of $M_4$ ($[900, 1000]$) does not share the same data point as the range predicate of $q'$ ($[100, 500]$), it has common data with the range predicate of $M_3$, which is a relevant model.
\end{example}

After the relevant pre-materialized models are determined, the execution strategy graph is constructed in the same way as \cite{hasani2018efficient} does. For example, in Example \ref{eg: relevant_models}, the graph is shown in Figure \ref{fig:min_cost_graph2}.

\begin{figure}[t]
\begin{center}
\begin{tikzpicture}[grow'=up,scale=0.75]
% \tikzset{vertex/.style = {shape=circle,draw,minimum size=1.5em}}
% \tikzset{edge/.style = {->,> = latex'}}
% vertices
\node (0) at (1,-2) {$100$};
\node (1) at (-2,0) {$250$};
\node (2) at (4,0) {$300$};
\node (3) at (-2,2) {$500$};
\node (4) at (1,4) {$900$};
\node (5) at (4,2) {$1000$};
%edges
% \draw[edge] (1) to (3);
% \path[->]
% (1) edge[bend left] node[swap] {$\neg$} (2)
% (2) edge[bend left] node {$\neg$} (1);
\scriptsize{
\draw[-] (1) --(3) ;%node[pos=0.5,left]{$C_{merge}([250, 500])$}; 
\draw[-] (1) --(2) ; 
\draw[-] (2) --(3) ; 
\draw[-] (3) --(5) ;
\draw[-] (2) --(4) ;
\draw[-] (4) --(5) ;
\draw[-] (3) --(4) ;
\draw[-] (1) --(5) ;
\draw[-] (2) --(5) ;
\draw[-] (1) --(4) ;
\draw[-] (0) --(1) ;
\draw[-] (0) --(2) ;
\draw[-] (0) --(3) ;
\draw[-] (0) --(4) ;
\draw[-] (0) --(5) ;
}
% \draw[edge] (2) to (3); 
% \draw [->] (-0.4,0.3) arc (10:330:20pt);
\end{tikzpicture}
\end{center}
\caption{Execution strategy graph in \cite{gupta2015processing}}
\label{fig:min_cost_graph2}
\end{figure}

Due to the space limit, the edge weights are not shown in Figure \ref{fig:min_cost_graph2}, which is computed similarly to compared to Figure \ref{fig:min_cost_graph}. Then Dijkstra's algorithm is applied to derive the shortest path from $100$ to $500$. Note that different from Figure \ref{fig:min_cost_graph}, the edges in Figure \ref{fig:min_cost_graph2} become undirected, which indicates that for an edge $i-j$ ($i < j$), the path traverse it by either direction. The traversal direction over an edge represents addition or removal of the model over the corresponding data points. For example, one potential minimal path $\mathcal{P}$ in Figure \ref{fig:min_cost_graph2} from $100$ to $500$ would be $100\rightarrow 300 \rightarrow 250 \rightarrow 500$ since $M_0$ and $M_1$ are built on the range predicate $[100, 300]$ and $[250, 500]$ respectively, which can be reused to reduce the cost. The data points within $[100, 300]$ and $[250, 500]$ are included for model construction for $q'$, which is also reflected in the direction of the path $\mathcal{P}$, i.e. $100 \rightarrow 300$ and $250 \rightarrow 500$. However, since there is an overlap between the two range predicates, the data points within $[250, 300]$ should be removed (see the edge direction $300\rightarrow250$ in the path $\mathcal{P}$), which is achieved by constructing another model from the scratch over $[250, 300]$ and combining it with $M_0$ and $M_1$ to remove the effect of the duplicated data points in $[250, 300]$.

\subsection{selecting models for pre-materialization}\label{sec: select_models}
The authors also explored how to select models for pre-materialization given a set of queries from a workload, which is resolved with two stages, i.e. {\em candidate generation step} to generate a set of possible models and {\em candidate selection step} to retain the best models from the previous step according to a utility metric.

In the {\em candidate generation step}, given a set of queries $\{Q_1, Q_2, \dots, Q_m\}$ and the corresponding predicates $\{[lb_1, ub_1], [lb_2, ub_2], \dots, [lb_m, ub_m]\}$, the candidate predicate $[l, u]$ is considered such that 1) $l \leq u$; 2) $l, u \in \{lb_1, ub_1, lb_2, ub_2, \dots, lb_m, ub_m\}$; 3) $[l, u]$ is included by at least one $[lb_i, ub_i]$ ($i=1,2,\dots, m$).

In the {\em candidate selection step}, the utility evaluation of a set of candidate pre-materialized models $\mathcal{M} = \{M_1, M_2, \dots, M_k\}$ is determined by the difference between the sum of the cost of answering query $Q_i$ ($i=1,2,\dots, m$) using $\mathcal{M}$ and the sum of the cost to evaluate every $Q_i$ without pre-materializing any models, in which the cost is computed by the shortest path in the graph as constructed in Section \ref{sec: opt}. Then $L$ models with highest utility are selected.

\subsection{For arbitrary queries}
The previous subsections only consider a type of queries with predicates over one attribute, which cannot represent the general cases in practice since the predicates of the queries may touch multiple attributes or hierarchical attributes, which can incur NP-hard time complexity. Given the cost measure proposed in Section \ref{sec: opt}, how to adapt the search algorithm for optimal strategy and how to select the models for pre-materialization for arbitrary queries are introduced below.

In terms of optimally selecting the pre-materialized models under the existence of multiple attributes in the query predicates, the selected pre-materialized models should cover the data points from $\textbf{D}$ touched by the query and incur minimal cost, which is an instantiation of set cover problem. To derive a set of optimal or nearly optimal pre-materialized models efficiently, a greedy algorithm is proposed, which originates from the greedy algorithm for the set cover problem and sketched as follows: Given a user query $q$, which retrieves a set of data points from $\textbf{D}$, $D_q$, a set of candidate pre-materialized models $\mathcal{M}_q$ can be determined. Then under the cost model proposed in Section \ref{sec: opt}, the model $M \in \mathcal{M}_q$ covering the most data points from $D_q$ with least cost is chosen and the data points covered by $M$ is removed from $D_q$, which proceeds iteratively until $D_q$ becomes empty.

In order to select models for pre-materialization for arbitrary queries, the {\em candidate generation step} and the {\em candidate selection step} are still needed. In the {\em candidate generation step}, all the pairs of $(Q_i, Q_j)$ from the workload which cover $D_i$ and $D_j$ from $\textbf{D}$ respectively are used to compose four candidate models over four set of data points respectively, i.e. $D_i, D_j, D_i \bigcup D_j, D_i \bigcap D_j$. Then same as Section \ref{sec: select_models}, in the {\em candidate selection step}, the $L$ models with the highest utility are chosen for pre-materialization.

% \subsection{Differences between \cite{hasani2018efficient} and \cite{gupta2015processing}}
% Both \cite{hasani2018efficient} and \cite{gupta2015processing} mainly deal with queries with range predicates over one attribute, which both determine the minimal-cost execution strategy by building the execution strategy graph with relevant pre-materialized models and the user query (introduced in Section \ref{sec: opt}) and then applying Dijkstra's algorithm over that. The differences between \cite{hasani2018efficient} and \cite{gupta2015processing} lie in the concept of relevant pre-materialized models for a given query and thus the construction of the execution strategy graph.
\input{Sections/Critique.tex}

\newpage

\bibliographystyle{abbrv}
\bibliography{bibliography}

\end{document}
